\newglossaryentry{N}{name=\ensuremath{\mathbb{N}}, description={Menge}}
\newglossaryentry{R}{name=\ensuremath{\mathbb{R}}, description={Menge der reellen Zahlen}}
\newglossaryentry{R+}{name=\ensuremath{\mathbb{R}^+}, description={Menge der positiven reellen Zahlen inklusive Null}}

\newglossaryentry{I}{name=\ma{I}, description={Identitätsmatrix}}

\newglossaryentry{diag}{name=\ensuremath{\mathrm{diag}}, description={Diagonalfunktion}}
\newglossaryentry{ortho}{name=\ensuremath{\perp}, description={Orthogonalität}}
\newglossaryentry{hadamard}{name=\ensuremath{\odot}, description={elementweises Hadamard-Produkt}}
\newglossaryentry{O}{name=\ensuremath{\mathcal{O}}, description={O-Notation}}
\newglossaryentry{T}{name=\ensuremath{T}, description={Tschebyschow-Polynom}}

\newglossaryentry{G}{name=\ensuremath{\mathcal{G}}, description={Graph}}
\newglossaryentry{V}{name=\ensuremath{\mathcal{V}}, description={Knotenmenge ${\left\{v_i\right\}}^N_{i=1}$ eines Graphen \gls{G}}}
\newglossaryentry{E}{name=\ensuremath{\mathcal{E}}, description={Kantenmenge}}
\newglossaryentry{v}{name={\ensuremath{v}}, description={Knoten eines Graphen}}
\newglossaryentry{A}{name=\ma{A}, description={Adjazentmatrix eines Graphen \gls{G}}}
\newglossaryentry{D}{name=\ma{D}, description={gewichtete Gradmatrix}}
\newglossaryentry{L}{name=\ma{L}, description={Laplacian, unnormalisiert}}
\newglossaryentry{Neighbor}{name=\ensuremath{\mathcal{N}}, description={Nachbarschaftsfunktion}}
\newglossaryentry{Lnorm}{name=\ma{\tilde{L}}, description={Laplacian, normalisiert}}
\newglossaryentry{Lboth}{name=\ma{\mathcal{L}}, description={Laplacian, normalisiert oder unnormalisiert}}
\newglossaryentry{w}{name=\ensuremath{w}, description={Gewichtsfunktion der Kanten eines Graph \gls{G} mit $\gls{w} \colon \gls{V} \times \gls{V} \to \gls{R+}$}}
\newglossaryentry{adj}{name=\ensuremath{\sim}, description={Adjazenzrelation zweiter Knoten eines Graphen \gls{G} mit $u \gls{adj} v$ genau dann, wenn $u$ und $v$ adjazent}}
\newglossaryentry{d}{name=\ensuremath{d}, description={gewichtete Gradfunktion der Knoten eines Graphen \gls{G} mit $\gls{d} \colon \gls{V} \to \gls{R+}$}}
\newglossaryentry{s}{name=\ensuremath{s}, description={kürzeste Pfaddistanz mit $s \colon \gls{V} \times \gls{V} \to \gls{N}$}}

\newglossaryentry{lambda}{name=\ensuremath{\lambda}, description={Eigenwert eines Eigenwertproblems $\ma{M}\ve{u} = \lambda\ve{u}$}}
\newglossaryentry{lambdamax}{name=\ensuremath{\lambda_{\max}}, description={Größter Eigenwert eines Eigenwertproblems $\ma{M}\ve{u} = \lambda\ve{u}$}}
\newglossaryentry{Lambda}{name=\ma{\Lambda}, description={Diagonalmatrix der Eigenwerte einer Matrix \ma{M} mit $\mathrm{diag}$}}
\newglossaryentry{eiv}{name=\ve{u}, description={normierter Eigenvektor zu einem Eigenwert mit $\left\|\gls{eiv}\right\|_2 = 1$}}
\newglossaryentry{Eiv}{name=\ma{U}, description={Eigenvektormatrix $\left[\ve{u}_1, \ldots, \ve{u}_N \right] \in \mathbb{R}^{N \times N}$ von $N$ Eigenvektoren $\ve{u}_i$}}
\newglossaryentry{Lambdatilde}{name=\ma{\tilde{\Lambda}}, description={reskalierte Diagonalmatrix der Eigenwerte des Laplacian}}
\newglossaryentry{Lbothtilde}{name=\ma{\tilde{\mathcal{L}}}, description={Laplacian reskaliert}}
\newglossaryentry{Atilde}{name=\ma{\tilde{A}}, description={reskalierte Diagonalmatrix der Eigenwerte des Laplacian}}
\newglossaryentry{Dtilde}{name=\ma{\tilde{D}}, description={reskalierte Diagonalmatrix der Eigenwerte des Laplacian}}
