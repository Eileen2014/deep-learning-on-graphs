\begin{figure}[t]
\centering
\subfigure[Gitterrepräsentation]{\begin{tikzpicture}
    \tikzstyle{node}=[circle,draw, minimum width=5pt, inner sep=0pt, fill=white]
    \tikzstyle{root}=[fill=black]

    \node[node, root] (root) at (0,0) {};
    \node[node, label={$\left(x-\delta, y\right)$}] (a) at (-2, 0) {};
    \node[node, label={$\left(x+\delta, y\right)$}] (b) at (2, 0) {};
    \node[node, label={[shift={(1.2, -0.22)}]$\left(x, y+\delta\right)$}] (c) at (0, -2) {};
    \node[node, label={[shift={(-1.2, -0.6)}]$\left(x, y-\delta\right)$}] (d) at (0, 2) {};

    \path (root) edge (a);
    \path (root) edge (b);
    \path (root) edge (c);
    \path (root) edge (d);
  \end{tikzpicture}
}
\hspace{1cm}
\subfigure[Graphrepräsentation]{\begin{tikzpicture}
    \tikzstyle{node}=[circle,draw, minimum width=5pt, inner sep=0pt, fill=white]
    \tikzstyle{root}=[fill=black]

    \node[node, root] (root) at (0,0) {};
    \node[node] (a) at (-2, 0) {};
    \node[node] (b) at (2, 0) {};
    \node[node] (c) at (0, -2) {};
    \node[node] (d) at (0, 2) {};

    \path (root) edge node[above] {$1/\delta^2$} (a);
    \path (root) edge node[below] {$1/\delta^2$} (b);
    \path (root) edge node[left] {$1/\delta^2$} (c);
    \path (root) edge node[right] {$1/\delta^2$} (d);
  \end{tikzpicture}
}
\caption[5-Punkte-Stern]{Illustration des 5-Punkte-Sterns in zwei Dimensionen mit gleicher Approximation des $\nabla^2$ Operators (bei umgekehrtem Vorzeichen), einmal mit der 5-Punkte-Stern Approximation auf regulären Gittern (a) und einmal mit der kombinatorischen Laplace-Matrix \gls{L} auf Graphen (b).}
\label{fig:5_punkte_stern}
\end{figure}
