\documentclass[pdftex,12pt,a4paper,twoside,ngerman,numbers=noenddot]{scrbook}

% -------------------------------------------------------------------

% Seitenformat anpassen
\usepackage[a4paper,left=3.5cm,right=2.5cm,bottom=3.5cm,top=3cm]{geometry}
\setlength{\headheight}{19pt}

% -------------------------------------------------------------------

% Fix für alte Pakete mit KOMA Warnung
\usepackage{scrhack}

% -------------------------------------------------------------------

% Absolute Positionierung für Titelseite
\usepackage[absolute,overlay]{textpos}
\setlength{\TPHorizModule}{1mm}
\setlength{\TPVertModule}{\TPHorizModule}
\textblockorigin{0mm}{0mm}
\usepackage{setspace}

% -------------------------------------------------------------------

% Schrifteinstellungen
\usepackage{lmodern}
\usepackage[english,main=ngerman]{babel}
\usepackage[utf8]{inputenc}
\usepackage[T1]{fontenc}
\usepackage{ae,aecompl}

% -------------------------------------------------------------------

% Bibtex deutsch
\usepackage[numbers,sort]{natbib}

% -------------------------------------------------------------------

% Anführungszeichen
\usepackage[babel,german=quotes]{csquotes}

% -------------------------------------------------------------------

% URLs
\usepackage{url}

% Trennung langer URLs
\usepackage[hyphenbreaks]{breakurl}
\def\UrlBreaks{\do\a\do\b\do\c\do\d\do\e\do\f\do\g\do\h\do\i\do\j\do\k\do\l%
\do\m\do\n\do\o\do\p\do\q\do\r\do\s\do\t\do\u\do\v\do\w\do\x\do\y\do\z\do\0%
\do\1\do\2\do\3\do\4\do\5\do\6\do\7\do\8\do\9\do\-}%

% -------------------------------------------------------------------

% Caption anpassen
\usepackage[margin=0pt,font=small,labelfont=bf]{caption}

% -------------------------------------------------------------------

% Zeilenabstand einstellen
\renewcommand{\baselinestretch}{1.25}

% Floating-Umgebungen anpassen
\renewcommand{\topfraction}{0.9}
\renewcommand{\bottomfraction}{0.8}

% -------------------------------------------------------------------

% Keine einzelnen Zeilen beim Anfang eines Abschnitts (Schusterjungen)
% \clubpenalty=10000
% Keine einzelnen Zeilen am Ende eines Abschnitts (Hurenkinder)
% \widowpenalty=10000
% \displaywidowpenalty=10000

% -------------------------------------------------------------------

% Keine Einrücktiefe der ersten Zeile eines neuen Absatzes.
\parindent=0cm

% -------------------------------------------------------------------

% Kopfzeile hinzufügen
\usepackage[headsepline]{scrlayer-scrpage}
\clearpairofpagestyles{}

\lehead{\pagemark}
\rehead{\headmark}
\rohead{\pagemark}
\lohead{\headmark}

% -------------------------------------------------------------------

% Eigene Farben
\usepackage{xcolor}
\definecolor{TUGreen}{rgb}{0.517,0.721,0.094}
\definecolor{red}{rgb}{1,0,0}

% -------------------------------------------------------------------

% Algorithmen
\usepackage[plain,chapter]{algorithm}
\usepackage{algorithmic}

% Algorithmen anpassen
\renewcommand{\algorithmicrequire}{\textit{Eingabe:}}
\renewcommand{\algorithmicensure}{\textit{Ausgabe:}}
\floatname{algorithm}{Algorithmus}
\renewcommand{\listalgorithmname}{Algorithmenverzeichnis}
\renewcommand{\algorithmiccomment}[1]{\color{grau}{// #1}}

% -------------------------------------------------------------------

% Grafikpakete einbinden
\usepackage{graphicx}
\usepackage{subfigure}
\usepackage{flafter}  % Platziere Figuren immer nach ihrer ersten Referenz
\usepackage{pdfpages}
\usepackage{tikz}

% -------------------------------------------------------------------

% Mathematikpakete einbinden

\usepackage{amsmath,amssymb,amsthm}
\usepackage{mathtools}

% -------------------------------------------------------------------

% Todonotes einbinden

\usepackage{todonotes}

% -------------------------------------------------------------------

% Glossar
\usepackage[toc]{glossaries}
\glstoctrue{}
\makeglossaries{}
\loadglsentries{glossaries}
\loadglsentries{acronyms}

% -------------------------------------------------------------------

% Fix für \left und \right Abstände
\let\originalleft\left
\let\originalright\right
\def\left#1{\mathopen{}\originalleft#1}
\def\right#1{\originalright#1\mathclose{}}

% -------------------------------------------------------------------

% Informationen für PDF-Dokument festlegen
\usepackage[pdfencoding=auto]{hyperref}

\hypersetup{pdfauthor={\Autor},
            pdftitle={\Titel},
            pdfsubject={\Arbeit, \Universitaet, \Fakultaet},
            pdfproducer={LaTeX},
            pdfview=FitV,
            pdfstartview=FitV,
            pdfhighlight=/I,
            pdfborder=0 0 0,
            colorlinks=false,
            bookmarksopen,
            bookmarksopenlevel=1,
            bookmarksnumbered=false,
            plainpages=false
}
