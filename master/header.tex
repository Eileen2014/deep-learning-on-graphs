\documentclass[pdftex,12pt,a4paper,twoside,ngerman,numbers=noenddot]{scrbook}

% -------------------------------------------------------------------

% Seitenformat anpassen
\usepackage[a4paper,left=3.5cm,right=2.5cm,bottom=3.5cm,top=3cm]{geometry}
\setlength{\headheight}{19pt}

% -------------------------------------------------------------------

% Fix für alte Pakete mit KOMA Warnung
\usepackage{scrhack}

% -------------------------------------------------------------------

% Absolute Positionierung für Titelseite
\usepackage[absolute,overlay]{textpos}
\setlength{\TPHorizModule}{1mm}
\setlength{\TPVertModule}{\TPHorizModule}
\textblockorigin{0mm}{0mm}
\usepackage{setspace}

% -------------------------------------------------------------------

% Schrifteinstellungen
\usepackage{lmodern}
\usepackage[english,main=ngerman]{babel}
\usepackage[utf8]{inputenc}
\usepackage[T1]{fontenc}
\usepackage{ae,aecompl}

% -------------------------------------------------------------------

% Bibtex deutsch
\usepackage[numbers,sort]{natbib}

% -------------------------------------------------------------------

% Anführungszeichen
\usepackage[babel,german=quotes]{csquotes}

% -------------------------------------------------------------------

% URLs
\usepackage{url}

% Trennung langer URLs
\usepackage[hyphenbreaks]{breakurl}
\def\UrlBreaks{\do\a\do\b\do\c\do\d\do\e\do\f\do\g\do\h\do\i\do\j\do\k\do\l%
\do\m\do\n\do\o\do\p\do\q\do\r\do\s\do\t\do\u\do\v\do\w\do\x\do\y\do\z\do\0%
\do\1\do\2\do\3\do\4\do\5\do\6\do\7\do\8\do\9\do\-}%

% -------------------------------------------------------------------

% Caption anpassen
\usepackage[margin=0pt,font=small,labelfont=bf]{caption}

% -------------------------------------------------------------------

% Tabellen
\usepackage{booktabs}  % bessere Linien
\usepackage{siunitx}  % Ausrichtung von Dezimalstellen

% -------------------------------------------------------------------

% Zeilenabstand einstellen
\renewcommand{\baselinestretch}{1.25}

% Floating-Umgebungen anpassen
\renewcommand{\topfraction}{0.9}
\renewcommand{\bottomfraction}{0.8}

% -------------------------------------------------------------------

% Keine Einrücktiefe der ersten Zeile eines neuen Absatzes.
\parindent=0cm

% -------------------------------------------------------------------

% Kopfzeile hinzufügen
\usepackage[headsepline]{scrlayer-scrpage}
\clearpairofpagestyles{}

\lehead{\pagemark}
\rehead{\headmark}
\rohead{\pagemark}
\lohead{\headmark}

% -------------------------------------------------------------------

% Eigene Farben
\usepackage{xcolor}
\definecolor{TUGreen}{rgb}{0.517,0.721,0.094}
\definecolor{red}{rgb}{1,0,0}

% -------------------------------------------------------------------

% Algorithmen
\usepackage[plain,chapter]{algorithm}
\usepackage{algorithmic}

% Algorithmen anpassen
\renewcommand{\algorithmicrequire}{\textit{Eingabe:}}
\renewcommand{\algorithmicensure}{\textit{Ausgabe:}}
\floatname{algorithm}{Algorithmus}
\renewcommand{\listalgorithmname}{Algorithmenverzeichnis}
\renewcommand{\algorithmiccomment}[1]{\color{grau}{// #1}}

% -------------------------------------------------------------------

% Grafikpakete einbinden
\usepackage{graphicx}
\usepackage{subfigure}
\usepackage{pdfpages}
\usepackage{tikz}
\usetikzlibrary{decorations.pathreplacing}

% -------------------------------------------------------------------

% Mathematikpakete einbinden

\usepackage{amsmath,amssymb,amsthm}
\usepackage{mathtools}

% -------------------------------------------------------------------

% Todonotes einbinden

\usepackage{todonotes}

% -------------------------------------------------------------------

% Glossar
\usepackage[toc,shortcuts]{glossaries}
\glstoctrue{}
\makeglossaries{}
\newglossaryentry{N}{name=\ensuremath{\mathbb{N}}, description={Menge}}
\newglossaryentry{R}{name=\ensuremath{\mathbb{R}}, description={Menge der reellen Zahlen}}
\newglossaryentry{R+}{name=\ensuremath{\mathbb{R}^+}, description={Menge der positiven reellen Zahlen inklusive Null}}

\newglossaryentry{I}{name=\ma{I}, description={Identitätsmatrix}}

\newglossaryentry{diag}{name=\ensuremath{\mathrm{diag}}, description={Diagonalfunktion}}
\newglossaryentry{ortho}{name=\ensuremath{\perp}, description={Orthogonalität}}
\newglossaryentry{hadamard}{name=\ensuremath{\odot}, description={elementweises Hadamard-Produkt}}
\newglossaryentry{O}{name=\ensuremath{\mathcal{O}}, description={O-Notation}}
\newglossaryentry{T}{name=\ensuremath{T}, description={Tschebyschow-Polynom}}

\newglossaryentry{G}{name=\ensuremath{\mathcal{G}}, description={Graph}}
\newglossaryentry{V}{name=\ensuremath{\mathcal{V}}, description={Knotenmenge ${\left\{v_i\right\}}^N_{i=1}$ eines Graphen \gls{G}}}
\newglossaryentry{v}{name={\ensuremath{v}}, description={Knoten eines Graphen}}
\newglossaryentry{A}{name=\ma{A}, description={Adjazentmatrix eines Graphen \gls{G}}}
\newglossaryentry{D}{name=\ma{D}, description={gewichtete Gradmatrix}}
\newglossaryentry{L}{name=\ma{L}, description={Laplacian, unnormalisiert}}
\newglossaryentry{Lnorm}{name=\ma{\tilde{L}}, description={Laplacian, normalisiert}}
\newglossaryentry{Lboth}{name=\ma{\mathcal{L}}, description={Laplacian, normalisiert oder unnormalisiert}}
\newglossaryentry{w}{name=\ensuremath{w}, description={Gewichtsfunktion der Kanten eines Graph \gls{G} mit $\gls{w} \colon \gls{V} \times \gls{V} \to \gls{R+}$}}
\newglossaryentry{adj}{name=\ensuremath{\sim}, description={Adjazenzrelation zweiter Knoten eines Graphen \gls{G} mit $u \gls{adj} v$ genau dann, wenn $u$ und $v$ adjazent}}
\newglossaryentry{d}{name=\ensuremath{d}, description={gewichtete Gradfunktion der Knoten eines Graphen \gls{G} mit $\gls{d} \colon \gls{V} \to \gls{R+}$}}
\newglossaryentry{s}{name=\ensuremath{s}, description={kürzeste Pfaddistanz mit $s \colon \gls{V} \times \gls{V} \to \gls{N}$}}

\newglossaryentry{lambda}{name=\ensuremath{\lambda}, description={Eigenwert eines Eigenwertproblems $\ma{M}\ve{u} = \lambda\ve{u}$}}
\newglossaryentry{lambdamax}{name=\ensuremath{\lambda_{\max}}, description={Größter Eigenwert eines Eigenwertproblems $\ma{M}\ve{u} = \lambda\ve{u}$}}
\newglossaryentry{Lambda}{name=\ma{\Lambda}, description={Diagonalmatrix der Eigenwerte einer Matrix \ma{M} mit $\mathrm{diag}$}}
\newglossaryentry{eiv}{name=\ve{u}, description={normierter Eigenvektor zu einem Eigenwert mit $\left\|\gls{eiv}\right\|_2 = 1$}}
\newglossaryentry{Eiv}{name=\ma{U}, description={Eigenvektormatrix $\left[\ve{u}_1, \ldots, \ve{u}_N \right] \in \mathbb{R}^{N \times N}$ von $N$ Eigenvektoren $\ve{u}_i$}}
\newglossaryentry{Lambdatilde}{name=\ma{\tilde{\Lambda}}, description={reskalierte Diagonalmatrix der Eigenwerte des Laplacian}}
\newglossaryentry{Atilde}{name=\ma{\tilde{A}}, description={reskalierte Diagonalmatrix der Eigenwerte des Laplacian}}
\newglossaryentry{Dtilde}{name=\ma{\tilde{D}}, description={reskalierte Diagonalmatrix der Eigenwerte des Laplacian}}

\newacronym[plural=CNNs, longplural={Convolutional Neural Networks}]{CNN}{CNN}{Convolutional Neural Network}
\newacronym[plural=GCNs, longplural={Graph Convolutional Networks}]{GCN}{GCN}{Graph Convolutional Network}
\newacronym{SLIC}{SLIC}{Simple Linear Iterative Clustering}
\newacronym{MNIST}{MNIST}{Modified National Institute of Standards and Technology}
\newacronym{Cifar}{CIFAR}{Canadian Institute for Advanced Research}
\newacronym{Pascal}{PASCAL VOC}{Pascal Visual Object Classes}
\newacronym{SVHN}{SVHN}{Street View House Numbers}
\newacronym{PCA}{PCA}{Haptkomponentenanalyse}


% -------------------------------------------------------------------

% Fix für \left und \right Abstände
\let\oldleft\left
\let\oldright\right
\def\left#1{\mathopen{}\oldleft#1}
\def\right#1{\oldright#1\mathclose{}}

% -------------------------------------------------------------------

% Füge einen Punkt an Paragraph-Überschriften
\let\oldparagraph=\paragraph
\renewcommand\paragraph[1]{\oldparagraph{#1.}}

% -------------------------------------------------------------------

% Informationen für PDF-Dokument festlegen
\usepackage[pdfencoding=auto]{hyperref}

\hypersetup{pdfauthor={\Autor},
            pdftitle={\Titel},
            pdfsubject={\Arbeit, \Universitaet, \Fakultaet},
            pdfproducer={LaTeX},
            pdfview=FitV,
            pdfstartview=FitV,
            pdfhighlight=/I,
            pdfborder=0 0 0,
            colorlinks=false,
            bookmarksopen,
            bookmarksopenlevel=1,
            bookmarksnumbered=false,
            plainpages=false
}
