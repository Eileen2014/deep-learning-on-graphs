\section{Superpixel}
\label{superpixel}

Eine Superpixelrepräsentation eines Bildes
Die Segmentierungsmaske $\gls{S} \in {\left\{1, \ldots, S\right\}}^{Y \times X}$ eines Bildes $\gls{B} \in {\left[0, 1\right]}^{Y \times X \times C}$, wobei $S \in \gls{N}$ die Anzahl an Segmenten beschreibt.
Massefunktion $\gls{m} \colon \gls{V} \to \gls{N}$
$\gls{G} = \left(\gls{V}, \gls{E}, \gls{p}, \gls{m}\right)$

\paragraph{Globale Normierung}
\label{globale_normierung}

\paragraph{Lokale Normierung}
\label{globale_normierung}

\subsection{Verfahren}
\label{superpixel_verfahren}

\paragraph{SLIC}
\label{slic}

\cite{slic}

\emph{Simple Linear Iterative Clustering} (SLIC)

\paragraph{Quickshift}
\label{quickshift}

\cite{quickshift}

\paragraph{Weitere Verfahren}
\label{weitere_superpixel_verfahren}

\cite{felzenszwalb}

\subsection{Adjazenzmatrixbestimmung}
\label{adjazenzmatrixbestimmung}

\subsection{Merkmalsextraktion}
\label{merkmalsextraktion}

Die Darstellung eines Bildes über einen Graphen \gls{G}, der aus einer Superpixelrepräsentation \gls{S} gewonnen wurde, besitzt in der Regel weitaus weniger Knoten im Gegensatz zu der reinen Darstellung des Bildes über eine Gitterrepräsentation.
Die Superpixel \bzw{} die Regionen der Segmentierungsmaske können jedoch die willkürlichsten Formen annehmen und besitzen dabei lediglich die Einschränkung, dass diese stets zusammenhängend sind und keinerlei Löcher besitzen.
Die Form eines Superpixels muss demnach bestmöglichst eingefangen \bzw{} beschrieben werden können — ein Prozess, der in der Bildverarbeitung als \emph{Merkmalsextraktion} bekannt ist~\cite{momente}.
Ein geeigntes Mittel zur Beschreibung einzelner Objekte in einem segmentierten Bild sind die \emph{Momente}, welche in nicht-zentrierte, translationsinvariante, skalierungsinvariante und rotationsinvariante Momente unterschieden werden~\cite{momente}.

\paragraph{Nicht-zentrierte Momente}
\label{nicht_zentrierte_momente}

Zu einer binären Segmentierungsmaske $\gls{S} \in {\left\{0, 1\right\}}^{Y \times X}$ sind die \emph{nicht-zentrierten Momente} vom Grad $\left(i+j\right)$, $i,j\in\gls{N}$, definiert als~\cite{momente}
\begin{equation*}
  \gls{M}_{ij} \coloneqq \sum_y^Y \sum_x^X y^i x^j \gls{S}_{yx}.
\end{equation*}
Obwohl der Grad eines Moments beliebig hoch gewählt werden kann, so reichen in der Praxis meist wenige Momente niedrigen Grades aus ($\le 3$), um eine Region hinreichend genau zu charakterisieren~\cite{momente}.
Bildeigenschaften, die durch nicht-zentrierte Momente beschrieben werden können, sind unter anderem dessen Fläche über $\gls{M}_{00}$ sowie dessen absoluter Schwerpunkt $\left\{\bar{x}, \bar{y}\right\} = \left\{ \gls{M}_{01}/\gls{M}_{00}, \gls{M}_{10}/\gls{M}_{00}\right\}$~\cite{momente}.
Nicht-zentrierte Momente sind aufgrund ihrer Berücksichtigung der Position einer Region im Bild meist unerwünscht, sie helfen aber für die weitere Definition von translationsinvarianten Momenten.

\paragraph{Translationsinvariante Momente}
\label{translationsinvariante_momente}

Mit Hilfe der absoluten Schwerpunktskoordinaten können die \emph{translationsinvarianten Momente} über
\begin{equation*}
  \gls{mu}_{ij} \coloneqq \sum_y^Y \sum_x^X {\left(y - \bar{y}\right)}^i {\left(x - \bar{x}\right)}^j \gls{S}_{yx}.
\end{equation*}
definiert werden~\cite{momente}.

\paragraph{Skalierungsinvariante Momente}
\label{skalierungsinvariante_zentrierte_momente}

\paragraph{Rotationsinvariante Momente}
\label{rotationsinvariante_momente}




\cite{Siedhoff}
