\section{Gitter}
\label{gitter}

Die einfachste Form einer Graphrepräsentation \gls{G} eines Bildes $\gls{B} \in \gls{R}^{H \times W \times C}$ ist die Repräsentation des Bildes über einen regulären Gittergraphen, denn dafür müssen keine Berechnungen am Bild vorgenommen werden.
Ein \emph{regulärer Gittergraph im zweidimensionalen euklidischen Raum} ist ein Graph $\gls{G} = \left(\gls{V}, \gls{E}, \gls{p}\right)$, der aus einem regulären Gitter gewonnen wurde und demnach genau $N \coloneqq H \times W$ Knoten enthält, \dhe{} einen Knoten für jedes Pixel in \gls{B}~\cite{Defferrard}.
Die Positionsfunktion der Knoten $\gls{p} \colon \gls{V} \to \gls{R}^2$ entspricht damit genau der Koordinate des korrespondierenden Pixels im Urpsprungsbild.
Sei dafür $\gls{v} \in \gls{V}$ der Knoten zu dem Bildpunkt an Position $\left(x,y\right)$.
Folglich gilt für die Position des Knotens $\gls{p}\left(\gls{v}\right) \coloneqq {\left[x,y\right]}^{\top}$.
Die örtlich um den Knoten $\gls{v} \in \gls{V}$ liegenden Knoten gelten dann auch im Gittergraph als benachbart und werden über eine Kante in \gls{E} verbunden.
Dabei unterscheidet man zwischen zwei frei wählbaren \emph{Konnektivitäten} des Graphen~\cite{Defferrard}.
Eine Konnektivität von $4$ bedeutet, dass ein Knoten (ohne Berücksichtigung der Randknoten) genau vier Nachbarn besitzt, die horizontal und vertikal zu im stehen.
Bei einer Konnektivität von $8$ gelten zusätzlich die vier diagonal zu ihm stehenden Knoten als Nachbarn und die Nachbarschaft wird folglich als ein $3 \times 3$ Fenster um den Knoten \gls{v} verstanden.

Analog zu den Daten der Farbkanäle an den einzelnen Pixeln des Bildes hängt auch an den Knoten über einer Merkmalsfunktion $f \colon \gls{V} \to \gls{R}^C$ \bzw{} einer Merkmalsmatrix $\gls{F} \in \gls{R}^{N \times C}$ diese Information mit $f\left(\gls{v}\right) \coloneqq \gls{B}_{{p\left(\gls{v}\right)}_2,{p\left(\gls{v}\right)}_1}$.

\paragraph{Effiziente Adjazenzbestimmung}
\label{adjazenzbestimmung_gitter}

Entgegen der Pixelanordnung in einem Bild ist die Anordnung der Knoten in einem Gittergraphen völlig irrelevant und kann willkürlich gewählt werden.
Ein Aufbau der Kantenmenge \gls{E} \bzw{} der korrespondierenden, ungewichteten Adjazenzmatrix \gls{A} kann jedoch besonders effizient gestaltet werden, wenn die Knoten zeilenweise entsprechend ihrer Bildkoordinate angeordnet werden.
Dafür wird das Bild \gls{B} zunächst an dessen Rändern um eine zusätzliche Spalte \bzw{} Zeile erweitert, \dhe{} $\gls{B} \in \gls{R}^{\left(H + 2\right) \times \left(W + 2\right) \times C}$.
Daraus ergeben sich $N \coloneqq \left(H + 2\right)\left(W + 2\right)$ viele Knoten eines Graphen, die die jeweiligen Gitterpunkte repräsentieren.
Ein Knoten $\gls{v}_i \in \gls{V}$, $W+3 < i < N - W - 3$, ist dann adjazent zu den Knoten $\gls{v}_j \in \gls{V}$ mit
\begin{equation*}
  j \in \left\{i-W-2, i-1, i+1, i+W+2\right\}
\end{equation*}
bei einer Konnektivität von $4$ \bzw{} mit
\begin{equation*}
  j \in \left\{i-W-3, i-W-2, i-W-1, i-1, i+1, i+W+1, i+W+2, i+W+3\right\}
\end{equation*}
bei einer Konnektivität von $8$.
Anschließend müssen die ungültigen Knoten, \dhe{} die an den Rändern des Gitters hinzugefügten Knoten, aus \gls{V} gelöscht werden.
Dazu zählen die ersten und letzten $W + 3$ Knoten aus \gls{V} sowie die vertikal am Rand des Gitters liegenden Knoten, die über eine Schrittweite von $W+2$ über der Knotenmenge eliminiert werden können.
