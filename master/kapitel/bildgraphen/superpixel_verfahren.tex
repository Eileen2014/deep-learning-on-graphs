\subsection{Verfahren}
\label{superpixel_verfahren}

\paragraph{SLIC}
\label{slic}

\emph{\gls{SLIC}}~\cite{slic}

Practically, we use the SLIC superpixels for their runtime and for their lower quantization error to decompose the image into superpixels~\cite{Gadde}.

\paragraph{Quickshift}
\label{quickshift}

\cite{quickshift}

\paragraph{Weitere Verfahren}
\label{weitere_superpixel_verfahren}

In der Literatur finden sich zahlreiche weitere Verfahren zur Bestimmung einer Superpixelrepräsentation aus einem Bild mit jeweils unterschiedlichen Stärken und Schwächen~\cite{super, slic}.
While the past few years have seen considerable progress in eigenvector-based
methods of image segmentation, these methods are too slow to be
practical for many applications.~\cite{felzenszwalb}.
While there are other approaches to image segmentation that are highly efficient, these
methods generally fail to capture perceptually important non-local properties of an
image~\cite{felzenszwalb}.
Namenshaft zu erwähnen sei hier noch die \emph{effiziente graphbasierte Bildsegmentierung} von Felzenszwalb, in der Regel unter dem Namen \emph{Felzenszwalb-Segmentierung} bekannt~\cite{felzenszwalb}.
As with certain classical clustering methods, our method is based on
selecting edges from a graph, where each pixel corresponds to a node in the graph,
and certain neighboring pixels are connected by undirected edges. Weights on each
edge measure the dissimilarity between pixels. However, unlike the classical methods,
our technique adaptively adjusts the segmentation criterion based on the degree of
variability in neighboring regions of the image~\cite{felzenszwalb}
