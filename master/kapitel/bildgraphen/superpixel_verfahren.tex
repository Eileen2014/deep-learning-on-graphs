\subsection{Verfahren}
\label{superpixel_verfahren}

\todo{was schreiben über viele Verfahren blabla, KURZ}

\paragraph{SLIC}
\label{slic}

Der Superpixelalgorithmus \emph{\gls{SLIC}} ist ein recht einfach gehaltener \emph{$K$-Means}-Clustering-Ansatz, welcher sich dennoch durch seine Geschwindigkeit, seine Speichereffizienz und insbesondere durch seine erfolgreiche Segmentierung hinsichtlich der Farbabgrenzen seiner Superpixel im Vergleich zu anderen \enquote{State-of-the-Art}-Algorithmen auszeichnet~\cite{slic, Gadde}.

Ein Bild $\gls{B} \in \gls{R}^{H \times C \times 3}$ im \emph{Lab-Farbraum} wird dazu initial in $K$ viele Clusterzentren $\ve{c}_k \coloneqq {\left[l_k, a_k, b_k, x_k, y_k\right]}^{\top}$  mit jeweils gleichmäßigem Abstand $S \coloneqq \sqrt{WH/K}$ zerlegt~\cite{slic}.
% Die Clusterzentren werden anschließend an die Position des geringsten Gradienten in ihrer $3 \times 3$ Nachbarschaft verschoben.
% Damit reduziert sich die Wahrscheinlichkeit, dass sich die Clusterzentren ${\left\{\ve{c}_k\right\}}_{k=1}^K$ auf einer Kante befinden~\cite{slic}.
Daraufhin werden in einem iterativen Prozess die Zugehörigkeiten der Pixel zu ihren jeweiligen Clustern über eine Distanzmetrik $D$ bestimmt und die Clusterzentren entsprechend ihrer neuen Gruppierungen angepasst.
Entgegen der konventionellen $k$-Means-Formulierung werden dabei aber nicht die Distanzen jedes Pixels zu jedem Clusterzentrum berechnet, sondern lediglich für die Pixel, die sich in der Region der Größe $2S \times 2S$ um $\ve{c}_k$ befinden~\cite{slic}.
Dies führt folglich zu einer drastischen Reduzierung der Anzahl an Distanzberechnungen und zu einem effizienteren Algorithmus.
Wurde zu jedem Pixel dessen ähnlichstes Clusterzentrum bestimmt, werden die Zentren über die Durchschnittsbildung all seiner zugehörigen Pixel neu justiert.
Nach einer festgelegten Anzahl an Iterationsschritten (üblicherweise $10$) bricht der Algorithmus schließlich ab~\cite{slic}.
Das gesamte Verfahren ist in Algorithmus~\ref{alg:slic} zusammengefasst.

\begin{algorithm}[t]
\centering
\begin{algorithmic}
  \REQUIRE{} Bild $\gls{B} \in \gls{R}^{H \times W \times 3}$ im Lab-Farbraum, Anzahl der Cluster $K$
  \ENSURE{} Segmentierungsmaske $\gls{Smaske} \in \gls{N}^{H \times W}$
  \STATE{} Initialisiere $K$ Clusterzentren ${\left\{\ve{c}_k\right\}}_{k=1}^K$ mit gleichmäßigem Abstand $S$.
  % \STATE{} Bewege Clusterzentren zu der Position des geringsten Gradienten in $3 \times 3$ Nachbarschaft.
  \STATE{} $\gls{Smaske}_{yx} \leftarrow -1$ für jedes Pixel in \gls{B} an $\left(x,y\right)$.
  \STATE{} $\ma{D}_{yx} \leftarrow \infty$ für jedes Pixel in \gls{B} an $\left(x,y\right)$.
  \REPEAT{}
    \FOR{$\ve{c}_k \in {\left\{\ve{c}_k\right\}}_{k=1}^K$}
      \FOR{Pixel in \gls{B} an $\left(x,y\right)$ in $2S \times 2S$ Region um $\ve{c}_k$}
        \STATE{} Berechne Distanz $D$ zwischen $\ve{c}_k$ und Pixel.
        \IF{$D < \ma{D}_{yx}$}
          \STATE{} $\ma{D}_{yx} \leftarrow D$
          \STATE{} $\gls{Smaske}_{yx} \leftarrow k$
        \ENDIF{}
      \ENDFOR{}
    \ENDFOR{}
    \STATE{} Justiere Clusterzentren ${\left\{\ve{c}_k\right\}}_{k=1}^K$.
  \UNTIL{Endbedingung}
\end{algorithmic}
\caption[\gls{SLIC}]{}
\label{alg:slic}
\end{algorithm}

Die Distanzmetrik $D$ basiert auf der euklidischen Norm ${\left\|\cdot\right\|}_2$ im fünfdimensionalen Raum ${\left[l,a,b,x,y\right]}^{\top}$ auf den Farben und den Positionen der Pixel \bzw{} der Clusterzentren.
Dabei ergeben sich jedoch Inkonsistenten für unterschiedliche Superpixelgrößen.
Bei großen Superpixeln führt dies zu einer stärkeren Gewichtung der räumlichen Positionen der Superpixel, wohingegen bei kleinen Superpixeln genau das Gegenteil der Fall ist~\cite{slic}.
Es ist daher notwendig, die Distanzen der Positionen $D_S$ und Farben $D_C$ getrennt voneinander zu betrachten und mittels dem Abstand $S$ der initialen Clusterzentren und einer Normalisierungskonstante $C \in \gls{R}$ \bzgl{} der Gwewichtung der Farben zu normieren.
Damit ergibt sich $D$ als~\cite{slic}
\begin{equation*}
\begin{split}
  D & \coloneqq \sqrt{{\left(\frac{D_S}{S}\right)}^2 + {\left(\frac{D_C}{C}\right)}^2}\\
  D_S & \coloneqq {\left\|{\left[x, y\right]}^{\top} - {\left[x_k, y_k\right]}^{\top}\right\|}_2\\
  D_C & \coloneqq {\left\|{\left[l, a, b\right]}^{\top} - {\left[l_k, a_k, b_k\right]}^{\top}\right\|}_2.
\end{split}
\end{equation*}
Die Normalisierungskonstante $C$ kann damit auch als Gewichtung zwischen der Form und den Farbabgrenzungen der Superpixel verstanden werden.
Falls $M$ sehr klein gewählt wird, respektieren die Superpixel Farbabgrenzungen besser, aber besitzen im Allgemeinen auch eine eher unreguläre Form~\cite{slic}.

Damit ist \gls{SLIC} ein $\gls{O}\left(WH\right)$ effizienter Algorithmus mit nur zwei frei wählbaren Parametern $K$ und $C$~\cite{slic}.
Viele Anwendungen im Bereich der Bildverarbeitung machen sich daher \gls{SLIC} zu Nutze (\vgl{}~\cite{Gadde, supercnn, super}).
Abbildung~\ref{fig:slic_quickshift} (a) zeigt zwei Resulte des \gls{SLIC}-Algorithmus bei einer unterschiedlich gewählten Anzahl an Clusterzentren.

\paragraph{Quickshift}
\label{quickshift}

\cite{quickshift}
Anwender:~\cite{Fulkerson}.

\paragraph{Weitere Verfahren}
\label{weitere_superpixel_verfahren}

In der Literatur finden sich zahlreiche weitere Verfahren zur Bestimmung einer Superpixelrepräsentation aus einem Bild mit jeweils unterschiedlichen Stärken und Schwächen~\cite{super, slic}.
While the past few years have seen considerable progress in eigenvector-based
methods of image segmentation, these methods are too slow to be
practical for many applications.~\cite{felzenszwalb}.
While there are other approaches to image segmentation that are highly efficient, these
methods generally fail to capture perceptually important non-local properties of an
image~\cite{felzenszwalb}.
Namenshaft zu erwähnen sei hier noch die \emph{effiziente graphbasierte Bildsegmentierung} von Felzenszwalb, die in der Regel unter dem Namen \emph{Felzenszwalb-Segmentierung} bekannt ist~\cite{felzenszwalb}.
As with certain classical clustering methods, our method is based on
selecting edges from a graph, where each pixel corresponds to a node in the graph,
and certain neighboring pixels are connected by undirected edges. Weights on each
edge measure the dissimilarity between pixels. However, unlike the classical methods,
our technique adaptively adjusts the segmentation criterion based on the degree of
variability in neighboring regions of the image~\cite{felzenszwalb}.
Die Felzenszwalb-Segementierung ist aber für die Anwendung auf einer Reihe von Bildern weniger geeignet, da dessen Parameter für jedes Bild eine spezielle Anpassung benötigen und folglich eine statischen Wahl dieser Parameter zu unbrauchbaren Ergebnissen führt.
