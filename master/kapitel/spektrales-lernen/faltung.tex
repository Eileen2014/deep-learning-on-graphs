\section{Spektraler Faltungsoperator}
\label{spektraler_faltungsoperator}

Sei $\ve{f} \in \gls{R}^N$ ein Signal auf den Knoten eines Graphen \gls{G}, welches abhängig von der Struktur des Graphen weiter verarbeitet werden soll.
Es ist jedoch in erster Linie nicht selbstverständlich, wie recht einfache, dennoch fundamentale Signalverarbeitungsprozesse wie Translation oder Filterung in der Domäne des Graphen definiert werden können~\cite{Shuman}.
\todo{das mit der translation} ohne translation keine faltung
Die spektrale Graphentheorie bietet einen Ausweg, in dem Eingabesignale in das Spektrum des Graphen zerlegt \bzw{} abgebildet und wieder retransformiert werden können.

\subsection{Graph-Fourier-Transformation}
\label{graph_fourier_transformation}

Das Spektrum eines Graphen \gls{G} bilden die Eigenwerte ${\left\{\gls{lambda}_i\right\}}_{i=1}^N$ der Laplace-Matrix \gls{Lboth} von \gls{G}.
Diese werden deshalb auch oft als die \emph{Frequenzen} von \gls{G} betitelt.
In der spektralen Domäne können wir ein Eingabeignal \ve{f} über \gls{G} dann analog wie ein zeitdiskretes Abtastsignal in der Fourier-Domäne behandeln.

\paragraph{Klassische Fourier-Transformation}
\label{klassische_fourier_transformation}

Die Fourier-Transformation $\hat f$ einer Funktion $f\left(t\right)$ ist definiert als~\cite{Shuman}
\begin{equation*}
  \hat f\left(\omega\right) \coloneqq \left\langle f, e^{2\pi i\omega t} \right\rangle = \int_{\gls{R}} f\left(t\right)e^{-2\pi i\omega t}\,\mathrm{d}t.
\end{equation*}
Die komplexen Exponentiale $e^{2\pi i\omega t}$ beschreiben dabei die Eigenfunktionen des eindimensionalen Laplace-Beltrami Operators~\cite{Shuman}
\begin{equation}
  - \nabla^2 e^{2\pi i\omega t} = - \frac{\partial^2}{\partial t^2} e^{2\pi i \omega t} = {\left(2\pi \omega\right)}^2 e^{2\pi i\omega t}.
  \label{eq:laplace_eigenfunktionen}
\end{equation}
$\hat f$ kann damit als die Ausdehnung von $f$ in Bezug auf die Eigenfunktionen des Laplace-Beltrami Operators $\nabla^2$ verstanden werden~\cite{Hammond}.
\\\\
Analog lässt sich die \emph{Graph-Fourier-Transformation} einer Funktion $f \colon \gls{V} \to \gls{R}$ \bzw{} $\ve{f} \in \gls{R}^N$ auf den Knoten eines Graphen \gls{G} als Ausdehnung von $f$ in Bezug auf die Eigenvektoren ${\left\{\gls{eiv}_i\right\}}_{i=1}^N$ der Laplace-Matrix \gls{Lboth} definieren~\cite{Shuman}:
\begin{equation}
  \hat f\left(\gls{lambda}_i\right) \coloneqq \left\langle \ve{f}, \gls{eiv}_i \right\rangle\, \text{\bzw{} } \ve{\hat f} \coloneqq \gls{Eiv}^{\top}\ve{f}.
  \label{eq:graph_fourier_transformation}
\end{equation}
Die inverse Graph-Fourier-Transformation ergibt sich dann als~\cite{Shuman}
\begin{equation}
  f\left(\gls{v}_i\right) = \sum_{j=1}^N \hat f\left(\gls{lambda}_j\right) {\left(\gls{eiv}_j\right)}_i\,\text{\bzw{} }\ve{f} = \gls{Eiv}\ve{\hat f}.
  \label{eq:inverse_graph_fourier_transformation}
\end{equation}

In der klassischen Fourier-Analyse sind für die Eigenwerte ${\left\{{\left(2\pi \omega\right)}^2\right\}}_{\omega \in \gls{R}}$ in~\eqref{eq:laplace_eigenfunktionen} nahe bei Null die korrespondieren Eigenfunktionen kleine, weich schwingende Funktionen, wohingegen für größere Eigenwerte \bzw{} Frequenzen die Eigenfunktionen sehr schnell und zügig anfangen zu oszillieren.
Bei der Graph-Fourier-Transformation ist dies ähnlich.
So ist für \gls{L} der erste Eigenvektor $\gls{eiv}_1 = \frac{1}{\sqrt{N}}{\left[1, \ldots, 1\right]}^{\top}$ zum Eigenwert $\gls{lambda}_1 = 0$ konstant und an jedem Knoten gleich.
Generell zeigt sich, dass die Eigenvektoren geringer Frequenzen nur geringfügig im Graph variieren, wohingegen Eigenvektoren größerer Eigenwerte immer unähnlicher werden (\vgl{}~\cite{Shuman}).
\\\\
Die Graph-Fourier-Transformation~\eqref{eq:graph_fourier_transformation} und ihre Inverse~\eqref{eq:inverse_graph_fourier_transformation} bieten uns eine Möglichkeit ein Signal in zwei unterschiedlichen Domänen zu repräsentieren, nämlich der Knotendomäne, \dhe{} das unveränderte Signal auf der Knotenmenge $f\left(\gls{v}_i\right)$, und der spektralen Domäne, \dhe{} das transformierte Signal in das Spektrum des Graphen $\hat f\left(\gls{lambda}_i\right)$.
Diese Transformation erlaubt uns die Formulierung fundamentaler Signalverarbeitungsoperationen.

\subsection{Spektrale Filterung}
\label{spektrale_filterung}

In der Signalverarbeitung versteht man unter der Frequenzfilterung die Transformation eines Eingabesignals in die Fourier-Domäne und der verstärkenden oder dämpfenden Veränderung der Amplituden der Frequenzkomponenten.
Formal betrachtet ergibt dies
\begin{equation}
  \hat f_{\mathrm{out}}\left(\omega\right) \coloneqq \hat f_{\mathrm{in}}\left(\omega\right)\hat g\left(\omega\right)
  \label{eq:fourier_filtering}
\end{equation}
mit Übertragungsfunktion $\hat g \colon \gls{R} \to \gls{R}$.
\citeauthor{Shuman} zeigen, dass die Filterung in der Fourier-Domäne äquivalent ist zu einer Faltung in der Zeitdomäne, \dhe{}
\begin{equation}
  \left(f_{\mathrm{in}} \star g\right)\left(t\right) \coloneqq \int_{\gls{R}} f_{\mathrm{in}}\left(\tau\right)g\left(t - \tau\right)\, \mathrm{d}\tau = f_{\mathrm{out}}\left(t\right)
  \label{eq:fourier_faltung}
\end{equation}

Wir können die Filterung der Frequenzen in der Fourier-Domäne analog zu~\eqref{eq:fourier_filtering} für die spektrale Domäne auf Graphen über
\begin{equation*}
  \hat f_{\mathrm{out}}\left(\gls{lambda}_i\right) \coloneqq \hat f_{\mathrm{in}}\left(\gls{lambda}_i\right)\hat g\left(\lambda_i\right)\,\text{\bzw{} }\ve{\hat f}_{\mathrm{out}} \coloneqq \ve{\hat f}_{\mathrm{in}} \gls{hadamard} \ve{\hat g}
\end{equation*}
beschreiben, wobei \gls{hadamard} das elementweise Hadamard-Produkt~\cite{Shuman}.
$\ve{\hat g} \in \gls{R}^N$ ist damit eine Übertragungsfunktion parametrisiert über die Eigenwerte ${\left\{\gls{lambda}_i\right\}}_{i=1}^N$ der Laplace-Matirx \gls{Lboth}.
Daraus ergibt sich analog zu~\eqref{eq:fourier_faltung} der \emph{spektrale Faltungsoperator} auf Graphen in der Knotendomäne mit Hilfe der Graph-Fourier-Transformation~\eqref{eq:graph_fourier_transformation} und ihrer Inversen~\eqref{eq:inverse_graph_fourier_transformation} als~\cite{Shuman, Defferrard}
\begin{equation}
  \ve{f}_{\mathrm{in}} \star \ve{\hat g} \coloneqq \gls{Eiv}\left(\gls{Eiv}^{\top}\ve{f}_{\mathrm{in}} \gls{hadamard} \ve{\hat g}\right) = \gls{Eiv} \hat g\left(\gls{Lambda}\right)\gls{Eiv}^{\top}\ve{f}_{\mathrm{in}} = \ve{f}_{\mathrm{out}},
  \label{eq:spektrale_faltung}
\end{equation}
mit $\hat g\left(\gls{Lambda}\right) = \gls{diag}\left({\left[\hat g\left(\gls{lambda}_1\right), \ldots, \hat g\left(\gls{lambda}_N\right)\right]}^{\top}\right)$.

$\hat g$ ist Filter!!\todo{hier erwähnen}

\subsection{Polynomielle Approximation}
\label{polynomielle_approximation}

Schwächen:
berechnungsteuer, Multiplikation mit der Eigenvektor Matrix U ist N2
Computing the eigendecomposition of L in the first place might be expensive too
Filtergröße bzw. Learning Complexity ist in O(n), der Dimensionalität der Eingangsdaten

\paragraph{Tschebyschow-Polynome}
\label{tschebyschow_polynome}
