\section{Graph Convolutional Networks}
\label{graph_convolutional_networks}

\citeauthor{gcn} motivieren einen weiteren Ansatz zur Faltung auf Graphen, genannt \emph{\gls{GCN}}, der auf der Methodik des spekralen Faltungsoperators aus Kapitel~\ref{spektraler_faltungsoperator} aufbaut und dabei wie eine \enquote{differenzierbare und parametriesierte Generalisierung des eindimensionalen Weisfeiler-Lehman Algorithmus auf Graphen} fungiert~\cite{gcn}.

\paragraph{Faltungsoperator}
\label{gcn_faltungsoperator}

Sei $\ve{f}_{\mathrm{in}} \star \ve{\hat g} \approx \sum_{k=0}^K c_k \gls{T}_k\left(\gls{Lbothtilde}\right) \ve{f}_{\mathrm{in}}$ der in~\eqref{eq:tschebyschow_faltung_L} definierte spektrale Faltungsoperator mit $K=1$.
Dann ist $\ve{f}_{\mathrm{in}} \star \ve{\hat g}$ eine lineare Funktion \bzgl{} \gls{Lboth} und damit eine lineare Funktion auf dem Spektrum des Graphen~\cite{gcn}.
Mit $K=1$ betrachtet der spektrale Faltungsoperator nur noch die lokale Nachbarschaft eines jeden Knoten (\vgl{}~\ref{polynomielle_approximation}).
Es ist anzumerken, dass dies in der Regel keinen Nachteil darstellt.
So hat es sich bei gegenwärtigen \enquote{State-of-the-Art}-\glspl{CNN} auf Bildern ebenfalls eingebürgert, nur noch über minimale $3\times3$ Filtergrößen zu falten und stattdessen Merkmale weit entfernterer Knoten über die mehrfache Aneinanderreihung der Faltungsschichten mittels tieferer Netze zu gewinnen~(\vgl{}~\cite{gcn, vgg, He}).
Unter dieser Restriktion vereinfacht sich $\ve{f}_{\mathrm{in}} \star \ve{\hat g}$ zu
\begin{equation}
  \ve{f}_{\mathrm{in}} \star \ve{\hat g} \approx c_0\, \ve{f}_{\mathrm{in}} + c_1 \left(\frac{2}{\gls{lambdamax}}\gls{Lboth} - \gls{I}\right)\ve{f}_{\mathrm{in}}
  \label{eq:gcn_faltung_both}
\end{equation}
mit zwei freien Parametern $c_0$ und $c_1$~\cite{gcn}.
Für $\gls{Lnorm}$ auf einem verbundenen Graphen \gls{G} gilt dann nach~\eqref{eq:gcn_faltung_both} weiter
\begin{equation}
  \ve{f}_{\mathrm{in}} \star \ve{\hat g} \approx c_0 \, \ve{f}_{\mathrm{in}} + c_1 \left(\gls{Lnorm} - \gls{I}\right)\ve{f}_{\mathrm{in}} = c_0 \, \ve{f}_{\mathrm{in}} - c_1 \gls{D}^{-\frac{1}{2}} \gls{A} \gls{D}^{-\frac{1}{2}} \ve{f}_{\mathrm{in}},
  \label{eq:gcn_faltung_norm}
\end{equation}
wobei $\gls{lambdamax} \coloneqq 2$ auf dessen oberste Schranke gesetzt wird~\cite{gcn}.
Um die Gefahr des Overfittings und die Anzahl an Berechnungen pro Schicht weiter zu beschränken, reduziert sich~\eqref{eq:gcn_faltung_norm} mit einem einzigen Parameter $c \coloneqq c_0$ mit $c = -c_1$ zu~\cite{gcn}
\begin{equation*}
  \ve{f}_{\mathrm{in}} \star \ve{\hat g} \approx c \left(\gls{I} + \gls{D}^{-\frac{1}{2}} \gls{A} \gls{D}^{-\frac{1}{2}} \right) \ve{f}_{\mathrm{in}}.
\end{equation*}
Die skalierten Eigenwerte von \gls{Lambdatilde} liegen auf Grund der Addition mit \gls{I} nun im Intervall $\left[0, 2\right]$ (\vgl{}~\cite{gcn}).
Demnach können wiederholte Anwendungen des Faltungsoperators zu \enquote{numerischen Instabilitäten und folglich zu explodierenden oder verschwindenen Gradienten} führen~\cite{gcn}.
\citeauthor{gcn} führen zur Behebung dieses Problems die folgende Renormalisierung durch: $\gls{I} + \gls{D}^{-1/2} \gls{A} \gls{D}^{-1/2} \rightarrow \gls{Dtilde}^{-1/2} \gls{Atilde} \gls{Dtilde}^{-1/2}$ mit $\gls{Atilde} \coloneqq \gls{A} + \gls{I}$ und $\gls{Dtilde}_{ii} \coloneqq \sum_{j=1}^N \gls{Atilde}_{ij}$.
Der entgültige Faltungsoperator des \glspl{GCN} ergibt sich dann als
\begin{equation}
  \ve{f}_{\mathrm{in}} \star \ve{\hat g} \approx c\, \gls{Dtilde}^{-\frac{1}{2}} \gls{Atilde} \gls{Dtilde}^{-\frac{1}{2}} \ve{f}_{\mathrm{in}}
  \label{eq:gcn_renorm}
\end{equation}
auf einem einzigen freien Parameter $c \in \gls{R}$.

\paragraph{Implementierung}
\label{gcn_tensor}

Die Faltung des \glspl{GCN} auf Merkmalsmatrizen lässt sich analog zur Tensorimplementierung des spektralen Faltungsoperators in Kapitel~\ref{polynomielle_approximation} beschreiben, mit dem Unterschied, dass wir aufgrund der Festlegung von $K=1$ keinen Gewichtstensor, sondern lediglich eine Gewichtsmatrix $\gls{W} \in \gls{R}^{M_{\mathrm{in}} \times M_{\mathrm{out}}}$ nutzen.
Die Faltung einer Eingabemerkmalsmatrix $\gls{F}_{\mathrm{in}} \in \gls{R}^{N \times M_{\mathrm{in}}}$ auf eine Ausgabemerkmalsmatrix $\gls{F}_{\mathrm{out}} \in \gls{R}^{N \times M_{\mathrm{out}}}$ ergibt sich dann als
\begin{equation*}
  \ma{F}_{\mathrm{out}} \coloneqq \gls{Dtilde}^{-\frac{1}{2}} \gls{Atilde} \gls{Dtilde}^{-\frac{1}{2}}\, \ma{F}_{\mathrm{in}} \gls{W}
\end{equation*}
mit Faltungsaufwand $\gls{O}\left(M_{\mathrm{in}}M_{\mathrm{out}}\left|\gls{E}\right|\right)$, weil $\gls{Atilde}\ma{F}_{\mathrm{in}}$ effizient mit der Multiplikation einer dünnbesetzten mit einer dichtbesetzten Matrix implementiert werden kann~\cite{gcn}.

\paragraph{Weisfeiler-Lehman Analogie}
\label{weisfeiler_lehman_analogie}
