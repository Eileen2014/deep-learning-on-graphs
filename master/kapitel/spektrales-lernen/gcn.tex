\section{Graph Convolutional Networks}
\label{graph_convolutional_networks}

\citeauthor{gcn} präsentieren einen weiteren Ansatz zur Faltung auf Graphen, genannt \emph{\gls{GCN}}, der auf der Methodik des spekralen Faltungsoperators aus Kapitel~\ref{spektraler_faltungsoperator} aufbaut und dabei wie eine \enquote{differenzierbare und parametrisierte Generalisierung des eindimensionalen Weisfeiler-Lehman-Algorithmus auf Graphen} fungiert~\cite{gcn}.

\paragraph{Faltungsoperator}
\label{gcn_faltungsoperator}

Sei $\gls{f}_{\mathrm{in}} \star \ve{\hat g} \approx \sum_{k=0}^K c_k \gls{T}_k\left(\gls{Lbothtilde}\right) \gls{f}_{\mathrm{in}}$ der in~\eqref{eq:tschebyschow_faltung_L} definierte spektrale Faltungsoperator mit $K=1$.
Dann ist $\gls{f}_{\mathrm{in}} \star \ve{\hat g}$ eine lineare Funktion \bzgl{} \gls{Lboth} und damit eine lineare Funktion auf dem Spektrum des Graphen~\cite{gcn}.
Mit $K=1$ betrachtet der spektrale Faltungsoperator nur noch die lokale Nachbarschaft eines jeden Knotens (\vgl{}~\ref{polynomielle_approximation}).
Es ist anzumerken, dass dies in der Regel keinen Nachteil darstellt.
So hat es sich bei gegenwärtigen \enquote{State-of-the-Art}-\glspl{CNN} auf Bildern ebenfalls bewährt, nur noch über minimale $3\times3$ Filtergrößen zu falten und stattdessen Merkmale weit entfernterer Knoten über die mehrfache Aneinanderreihung der Faltungsschichten mittels tieferer Netze zu gewinnen~(\vgl{}~\cite{gcn, vgg, He}).
Unter dieser Restriktion vereinfacht sich $\gls{f}_{\mathrm{in}} \star \ve{\hat g}$ zu
\begin{equation}
  \gls{f}_{\mathrm{in}} \star \ve{\hat g} \approx c_0\, \gls{f}_{\mathrm{in}} + c_1 \left(\frac{2}{\gls{lambdamax}}\gls{Lboth} - \gls{I}\right)\gls{f}_{\mathrm{in}}
  \label{eq:gcn_faltung_both}
\end{equation}
mit zwei freien Parametern $c_0$ und $c_1$~\cite{gcn}.
Für $\gls{Lnorm}$ auf einem zusammenhängenden Graphen \gls{G} gilt dann nach~\eqref{eq:laplace_norm_connected} und~\eqref{eq:gcn_faltung_both} weiter
\begin{equation}
  \gls{f}_{\mathrm{in}} \star \ve{\hat g} \approx c_0 \, \gls{f}_{\mathrm{in}} + c_1 \left(\gls{Lnorm} - \gls{I}\right)\gls{f}_{\mathrm{in}} = c_0 \, \gls{f}_{\mathrm{in}} - c_1 \gls{D}^{-\frac{1}{2}} \gls{A} \gls{D}^{-\frac{1}{2}} \gls{f}_{\mathrm{in}},
  \label{eq:gcn_faltung_norm}
\end{equation}
wobei $\gls{lambdamax} \coloneqq 2$ auf dessen obere Schranke gesetzt wird~\cite{gcn}.
Um die Gefahr des Overfittings und die Anzahl an Berechnungen pro Schicht weiter zu beschränken, reduziert sich~\eqref{eq:gcn_faltung_norm} mit einem einzigen Parameter $c \coloneqq c_0$ mit $c = -c_1$ zu~\cite{gcn}
\begin{equation*}
  \gls{f}_{\mathrm{in}} \star \ve{\hat g} \approx c \left(\gls{I} + \gls{D}^{-\frac{1}{2}} \gls{A} \gls{D}^{-\frac{1}{2}} \right) \gls{f}_{\mathrm{in}}.
\end{equation*}
Die skalierten Eigenwerte von \gls{Lambdatilde} liegen aufgrund der Addition mit \gls{I} nun im Intervall $\left[0, 2\right]$ (\vgl{}~\cite{gcn}).
Demnach können wiederholte Anwendungen des Faltungsoperators zu \enquote{numerischen Instabilitäten und folglich zu explodierenden oder verschwindenen Gradienten} führen~\cite{gcn}.
\citeauthor{gcn} führen zur Behebung dieses Problems die folgende Renormalisierung durch: $\gls{I} + \gls{D}^{-1/2} \gls{A} \gls{D}^{-1/2} \rightarrow \gls{Dtilde}^{-1/2} \gls{Atilde} \gls{Dtilde}^{-1/2}$ mit $\gls{Atilde} \coloneqq \gls{A} + \gls{I}$ und $\gls{Dtilde}_{ii} \coloneqq \sum_{j=1}^N \gls{Atilde}_{ij}$ \bzw{} $\gls{Dtilde}_{ii} = \gls{D} + \gls{I}$.
Der entgültige Faltungsoperator des \glspl{GCN} ergibt sich dann als
\begin{equation}
  \gls{f}_{\mathrm{in}} \star \ve{\hat g} \approx c\, \gls{Dtilde}^{-\frac{1}{2}} \gls{Atilde} \gls{Dtilde}^{-\frac{1}{2}} \gls{f}_{\mathrm{in}}
  \label{eq:gcn_renorm}
\end{equation}
auf einem einzigen freien Parameter $c \in \gls{R}$.

\paragraph{Implementierung}
\label{gcn_tensor}

Die Faltung des \glspl{GCN} auf Merkmalsmatrizen lässt sich analog zur Tensorimplementierung des spektralen Faltungsoperators in Kapitel~\ref{polynomielle_approximation} beschreiben, mit dem Unterschied, dass wir aufgrund der Festlegung von $K=1$ keinen Filtertensor, sondern lediglich eine Filtermatrix $\gls{W} \in \gls{R}^{M_{\mathrm{in}} \times M_{\mathrm{out}}}$ nutzen.
Die Faltung einer Eingabemerkmalsmatrix $\gls{F}_{\mathrm{in}} \in \gls{R}^{N \times M_{\mathrm{in}}}$ auf eine Ausgabemerkmalsmatrix $\gls{F}_{\mathrm{out}} \in \gls{R}^{N \times M_{\mathrm{out}}}$ ergibt sich dann als
\begin{equation}
  \gls{F}_{\mathrm{out}} \coloneqq \gls{Dtilde}^{-\frac{1}{2}} \gls{Atilde} \gls{Dtilde}^{-\frac{1}{2}}\, \gls{F}_{\mathrm{in}} \gls{W}
  \label{eq:gcn_tensor}
\end{equation}
mit Faltungsaufwand $\gls{O}\left(M_{\mathrm{in}}M_{\mathrm{out}}\left|\gls{E}\right|\right)$, weil $\gls{Atilde}\gls{F}_{\mathrm{in}}$ effizient mit der Multiplikation einer dünnbesetzten mit einer dichtbesetzten Matrix implementiert werden kann~\cite{gcn}.

\paragraph{Beziehung zum Weisfeiler-Lehman-Algorithmus}
\label{weisfeiler_lehman_beziehung}

\begin{algorithm}[t]
\centering
\begin{algorithmic}
  \REQUIRE{} Initiale Knotenfärbung $\gls{f}^{\left(0\right)} \in \gls{R}^N$
  \ENSURE{} Finale Knotenfärbung $\gls{f}^{\left(T\right)} \in \gls{R}^N$ nach $T$ Durchläufen
  \STATE{} $t \leftarrow 0$
  \REPEAT{}
    \FOR{$\gls{v}_i \in \gls{V}$}
      \STATE{} $\gls{f}^{\left(t+1\right)}_i \leftarrow \mathrm{hash}\left(\sum_{\gls{v}_j \in \gls{Neighbor}\left(\gls{v}_i\right)} \gls{f}^{\left(t\right)}_j\right)$
    \ENDFOR{}
    \STATE{} $t \leftarrow t + 1$
  \UNTIL{Konvergenz}
  \RETURN{$\gls{f}^{\left(T\right)}$}
\end{algorithmic}
  \caption[Weisfeiler-Lehman]{Eindimensionaler Weisfeiler-Lehman-Algorithmus auf einer initialen Knotenfärbung \bzw{} Merkmalsfunktion $\gls{f}^{\left(0\right)} \in \gls{R}^N$ eines Graphen \gls{G} mit $\gls{v}_i \in \gls{Neighbor}\left(\gls{v}_i\right)$~\cite{wl}. Der Prozess der Verfärbung eines jeden Knotens $\gls{v}_i$ auf Basis der Farben seiner lokalen Nachbarknoten wird solange wiederholt, bis diese konvergieren.}
\label{alg:wl}
\end{algorithm}

Der \emph{eindimensionale Weis\-fei\-ler-Lehman Algorithmus} beschreibt eine weit untersuchte Methode zur Knotenklassifizierung eines Graphen basierend auf einer initialen Färbung \bzw{} Merkmalsfunktion auf den Knoten eines Graphen \gls{G}, die unter anderem zur Bestimmung von Graphisomorphismen genutzt wird~\cite{douglas}.
Basierend auf einer initialen kontinuierlichen Knotenfärbung $\gls{f}^{\left(0\right)} \in \gls{R}^N$ wird die Farbe eines jeden Knotens $\gls{v}_i \in \gls{V}$ sukzessive mit Hilfe einer Hashfunktion $\mathrm{hash}\left(\cdot\right)$ so angepasst, dass sie die vorangegangene Farbe des Knotens zusammen mit den Farben seiner lokalen Nachbarschaft repräsentiert.
Dieser Prozess wiederholt sich solange, bis eine stabile Knotenfärbung gefunden wurde, \dhe{} die gefundene Färbung des Graphen konvergiert (\vgl{} Algorithmus~\ref{alg:wl}).

Sei die Hashfunktion nun gegeben als eine differenzierbare, nicht-lineare Aktivierungsfunktion $\gls{act} \colon \gls{R} \to \gls{R}$, \bspw{} $\gls{relu}\left(\cdot\right) \coloneqq \max\left(\cdot, 0\right)$, eines neuronalen Netzes.
Dann ergibt sich eine Faltungsschicht des \glspl{GCN} durch die Verkettung des Faltungsoperators mit anschließender Aktivierung als
\begin{equation*}
  \gls{f}^{\left(t+1\right)}_i = \gls{act}\left(\sum_{\gls{v}_j \in \gls{Neighbor}\left(\gls{v}_i\right)} \frac{\gls{w}\left(\gls{v}_i, \gls{v}_j\right)}{\sqrt{d_i d_j}} \gls{f}^{\left(t\right)}_j \gls{W}^{\left(t\right)} \right),
\end{equation*}
wobei $\gls{w}\left(\gls{v}_i, \gls{v}_j\right)/\sqrt{d_i d_j} \in \gls{R}$ die Normalisierungskonstante für die Kante $\left(\gls{v}_i, \gls{v}_j\right) \in \gls{E}$ entsprechend der Normalisierung $\gls{Dtilde}^{-1/2}\gls{Atilde}\gls{Dtilde}^{-1/2}$ aus~\eqref{eq:gcn_renorm} und $\gls{W}^{\left(t\right)} \in \gls{R}^{N \times N}$ die Filtermatrix der $t$-en Faltungsschicht beschreibt~\cite{gcn}.
Folglich kann die Faltung des \glspl{GCN} als \enquote{differenzierbare und parametrisierte Generalisierung des eindimensionalen Weisfeiler-Lehman-Algorithmus auf Graphen} verstanden werden~\cite{gcn}.
