\section{Netzarchitektur}
\label{spektrale_netzarchitektur}

Die Architektur eines neuronalen Netzes auf Graphen mit spektralen Faltungen verhält sich durch die ähnliche Formulierung einer Faltungs- und einer Poo\-ling\-sch\-icht analog zu der Netzarchitektur klassischer \glspl{CNN}.
Dabei werden wie gewohnt mehrere Faltungsschichten hintereinander gestapelt und über teilweise gesetzte Poolingschichten die Anzahl der zu betrachtenden Knoten sukzessive reduziert, wobei sich die Anzahl der Merkmale dabei stetig erhöht.
Im Anschluss darauf finden sich dann für gewöhnlich zwei bis drei vollverbundene Schichten, die die Merkmalsgröße dann schlussendlich auf die gewünschte Ausgabegröße reduziert (\vgl{}~\cite{Nielsen}).

\glspl{CNN} auf Bildern erfordern dabei eine feste Eingabegröße.
Dafür werden die Bildermengen in der Regel insofern skaliert und zugeschnitten, dass diese alle die gleiche Bildgröße besitzen (\zB{} $224 \times 224$)~\cite{spp}.
Auf einer Menge von Graphen erscheint es schwierig, diese so anzupassen, dass diese alle die gleiche Anzahl an Knoten aufweisen.
So ist es zwar vorstellbar, zusätzliche \enquote{Fake}-Knoten zu einem Graphen hinzufügen, sodass diese alle eine feste Anzahl an Knoten aufweisen.
Neben dem erhöhtem Speicheraufwand ist dieser Ansatz nicht geeignet für unbekannte Graphen, die in das Netz eingespeist werden.
So können diese \evtl{} eine größere Anzahl als die zuvor festgelegte Größe aufweisen.
Ebenso liefert uns der Prozess der Graphvergröberung eine stets unterschiedliche Rerpräsentation eines Graphen, die wohlmöglich eine größere Menge an \enquote{Fake}-Knoten erfordert und damit die Größe eines Graphen in die Höhe treibt.
Es ist weiterhin schwierig, einen Graphen auf eine feste Größe zuzuschneiden.
So lässt es sich nie mit Sicherheit sagen, welche Knoten aus dem Graphen entfernt oder zusammengefasst werden können.
Die Architektur eines neuronalen Netzes auf Graphen erfordert deshalb eine Struktur, die auf dynamische Eingabegrößen funktioniert.

Alternative Architekturen, auch im Hinblick auf die Ersetzungsmöglichkeiten \bzgl{} des Average-Poolings, werden in Kapitel~\ref{ausblick} diskutiert.
