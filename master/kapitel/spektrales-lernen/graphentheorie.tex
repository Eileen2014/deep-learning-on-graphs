\section{Spektrale Graphentheorie}
\label{spektrale_graphentheorie}

Es gibt 3 große Quellen hier:
\begin{itemize}
  \item Spectral Graph Theory by Chung
  \item Discrete Laplace-Beltrami Operator
  \item Spectral Graph Theory by Spielman
\end{itemize}
+ 4 zum Lernen:
\begin{itemize}
  \item Semi Supervised Classification
  \item Fast Localized Spectral Filterung
  \item Wavelets on Graphs via Spectral Graph Theory
  \item The Emerging Field of Signal Processing on Graphs
  \item How powerful are Graph Convolutions? (Review)
\end{itemize}

\subsection{Eigenwerte und Eigenvektoren reell symmetrischer Matrizen}
\label{eigenwerte_symmetrischer_matrizen}

\todo{intro}

$\ma{M} \in \gls{R}^{N \times N}$.
$\gls{eiv} \in \gls{R}^{N}$, $\gls{eiv} \neq \mathbf{0}$.
$\gls{lambda} \in \gls{R}$.
\emph{Eigenwertproblem} $\ma{M}\gls{eiv} = \gls{lambda}\gls{eiv}$.
Zu einem \emph{Eigenwert} $\gls{lambda}$ gibt es unendlich viele (skalierte) \emph{Eigenvektoren} \gls{eiv}.
Wir definieren den Eigenvektor \gls{eiv} eines Eigenwertes \gls{lambda} daher eindeutig über die Bedingung $\left\|\gls{eiv}\right\|_2 = 1$.
Sei \ma{M} weiterhin reell symmetrisch, \dhe{} $\ma{M} = \ma{M}^{\top}$.
Dann gilt für zwei unterschiedliche Eigenvektoren $\gls{eiv}_1$ und $\gls{eiv}_2$, dass $\gls{eiv}_1 \gls{ortho} \gls{eiv}_2$.
Weiterhin hat \ma{M} genau $N$ reelle Eigenwerte mit ${\left\{\gls{lambda}_i\right\}}_{i=1}^N$.

Wir definieren zu \ma{M} die orthogonale \emph{Eigenvektormatrix} $\gls{Eiv} = \left[\gls{eiv}_1, \ldots, \gls{eiv}_n\right] \in \gls{R}^{N \times N}$, wobei $\gls{Eiv}\gls{Eiv}^{\top}=\gls{I}$, und dessen korrespondierende Eigenwertdiagonalmatrix $\gls{Lambda} = \gls{diag}\left(\left[\gls{lambda}_1, \ldots, \gls{lambda}_N\right]\right)$, \dhe{} $\gls{Lambda}_{ii} = \gls{lambda}_i$.
Dann gilt $\ma{M}\gls{Eiv} = \gls{Eiv}\gls{Lambda}$ und insbesondere ist \ma{M} diagonalisierbar über
\begin{equation}
  \ma{M} = \ma{M}\gls{Eiv}\gls{Eiv}^{\top} = \gls{Eiv}\gls{Lambda}\gls{Eiv}^{\top}.
\end{equation}

Weiterhin gilt für die $k$te Potenz von $\ma{M}$, $k \in \gls{N}$,
\begin{equation}
  \ma{M}^k = {\left(\gls{Eiv}\gls{Lambda}\gls{Eiv}^{\top}\right)}^k = \gls{Eiv}\gls{Lambda}^k\gls{Eiv}^{\top}.
\end{equation}

Dieser Zusammenhang lässt sicht leicht zeigen, wenn man die Potenz ausschreibt:
\begin{equation*}
  {\left(\gls{Eiv}\gls{Lambda}\gls{Eiv}^{\top}\right)}^k = \gls{Eiv}\gls{Lambda}\gls{Eiv}^{\top}\gls{Eiv}\gls{Lambda}\gls{Eiv}^{\top}\prod^{k-2}_{i=1} \gls{Eiv}\gls{Lambda}\gls{Eiv}^{\top} = \gls{Eiv}\gls{Lambda}^2\gls{Eiv}^{\top} \prod^{k-2}_{i=1} \gls{Eiv}\gls{Lambda}\gls{Eiv}^{\top} = \gls{Eiv}\gls{Lambda}^k \gls{Eiv}^{\top}
\end{equation*}

Falls \ma{M} weiterhin \emph{schwach diagonaldominant} ist, \dhe{}
\begin{equation}
  \sum_{j=1}^N \left|\ma{M}_{ij}\right| \leq \left|\ma{M}\right|_{ii}\text{ für alle }i \in \left\{1, \ldots, N\right\},
\end{equation}
sind ihre Eigenwerte $\lambda_i \in \gls{R+}$ positiv reell und es lässt sich auf diesen eine Ordnung definieren mit $0 \leq \gls{lambda}_1 \leq \cdots \leq \gls{lambda}_N$.
Insbesondere ist \ma{M} damit \emph{positiv-semidefinit}, \dhe{} $\ve{x}^{\top}\ma{M}\ve{x} \geq 0$ für alle $\ve{x} \in \gls{R}^{N}$.

\todo{quelle}

\subsection{Laplace-Matrix}
\label{laplace_matrix}

Our eigenvalues relate well to other graph invariants for general graphs in a way that other definitions (such as the eigenvalues of adjacency matrices) often fail to do.
The advantages of this definition are perhaps due to the fact that it is consistent with the eigenvalues in spectral geometry and in stochastic processes.
Many results which were only known for regular graphs can be generalized to all graphs~\cite{Chung}.

\todo{intro}

Für einen schleifenlosen, ungerichteteten, gewichtet oder ungewichteten Graphen \gls{G} und dessen Adjazenzmatrix \gls{A} mit Gradmatrix \gls{D} ist die \emph{kombinatorische Laplace-Matrix} \gls{L} definiert als $\gls{L} = \gls{D} - \gls{A}$~\cite{Chung}.
Die \emph{normalisierte Laplace-Matrix} \gls{Lnorm} ist definiert als $\gls{Lnorm} = \gls{D}^{-\frac{1}{2}} \gls{L} \gls{D}^{-\frac{1}{2}}$ mit der Konvention, dass $\gls{D}^{-\frac{1}{2}}_{ii} = 0$ für isolierte Knoten $\gls{v}_i \in \gls{V}$ in \gls{G}, \dhe{} $\gls{D}_{ii} = 0$~\cite{Chung}.
Daraus ergibt sich die elementweise Definition
\begin{equation*}
  \gls{Lnorm}_{ij} = \begin{cases}
  1, & \text{wenn }i = j,\\
    -\frac{\gls{w}\left(\gls{v}_i, \gls{v}_j\right)}{\sqrt{\gls{d}\left(\gls{v}_i\right)\gls{d}\left(\gls{v}_j\right)}}, & \text{wenn }\gls{v}_i \gls{adj} \gls{v}_j,\\
  0, & \text{sonst.}
\end{cases}
\end{equation*}
Für verbundene Graphen kann \gls{Lnorm} vereinfacht werden zu $\gls{Lnorm} = \gls{I} - \gls{D}^{-\frac{1}{2}} \gls{A} \gls{D}^{-\frac{1}{2}}$~\cite{Chung}.
Jeder Eintrag auf der Diagonalen der normalisierten Laplace-Matrix ist folglich $1$.
\gls{Lnorm} ist damit normalisiert auf den (gewichteten) Grad zweier adjazenter Knoten $\gls{v}_i$ und $\gls{v}_j$.
Es ist anzumerken, dass \gls{L} und insbesondere \gls{Lnorm} reell symmetrisch sind~\cite{Chung}, wohingegen eine Normalisierung der Form $\gls{D}^{-1}\gls{L}$ dies in der Regel nicht ist~\cite{Reuter}.

\gls{L} und \gls{Lnorm} sind keine ähnlichen Matrizen.
Insbesondere sind ihre Eigenvektoren unterschiedlich.
Die Nutzung von \gls{L} oder \gls{Lnorm} ist damit abhängig von dem Problem, welches man betrachtet~\cite{Hammond}.
Wir schreiben \gls{Lboth} wenn die Wahl der Laplace-Matrix, \gls{L} oder \gls{Lnorm}, für die weitere Berechnung zwar fest, aber irrelevant ist.

\paragraph{Visuelle Interpretation}
\label{laplace_interpretation}

\todo{ein festes todo}

\paragraph{Eigenschaften}
\label{laplace_eigenschaften}

\gls{Lboth} ist eine reell symmetrische, schwach diagonaldominante Matrix und damit insbesondere positiv semidefinit.
\todo{nur kombinatorisch ist schwach diagonaldominant}

$\gls{Lboth} \in \gls{R}^{n \times n}$ hat genau $n$ Eigenwerte ${\left\{\gls{lambda}_i\right\}}_{i = 1}^n \in \gls{R+}$ mit $\gls{lambda}_i \leq \gls{lambda}_{i+1}$.

Anzahl der Eigenvektoren gleich Null ist die Anzahl an Komponenten, die ein Graph besitzt.

Insbesondere sind jede Reihen- und Spaltensumme von $\gls{L}$ ist $0$, d.h.\ $\sum_j \gls{Lboth}_{ij} = 0$ und $\sum_j \gls{Lboth}_{ji} = 0$ für alle $i \in \left\{1, \ldots, n\right\}$.\todo{gilt nur für kombinatorischen!}
Insbesondere gilt $\gls{lambda}_1 = 0$, da $\gls{eiv}_1 = \frac{1}{\sqrt{n}}{\left[1, \ldots, 1\right]}^{\top} \in \gls{R}^n$ Eigenvektor von \gls{Lboth} mit $\gls{Lboth}\gls{eiv}_1 = 0$.\\

$0 = \gls{lambda}_1 < \gls{lambda}_2 \leq \cdots \leq \gls{lambda}_n$ wenn Graph verbunden.\todo{quelle}\\

\todo{was sagt $\gls{lambda}_2$ aus?}
Für einen Graphen \gls{G} definieren wir $\gls{lambda}_{\gls{G}} := \gls{lambda}_2$ und $\gls{lambda}_{\max} := \gls{lambda}_n$
Für \gls{Lnorm} gilt $\gls{lambda}_{\max} \leq 2$\todo{quelle}

Für $\gls{Lboth}^{k}$ mit $k \in \gls{N}$ gilt ${\left(\gls{Lboth}^k\right)}_{ij} = 0$ genau dann, wenn $\gls{s}\left(v_i, v_j\right) > k$~\cite{Hammond}.
Damit beschreibt $\gls{Lboth}^k$ bildlich gesprochen die Menge an Knoten, die maximal $k$ Kanten entfernt liegen.



laplace matrix ist reell symmetrisch und schwach diagonaldominant
