\section{Spektrale Graphentheorie}
\label{spektrale_graphentheorie}

Die spektrale Graphentheorie beschäftigt sich mit der Konstruktion, Analyse und Manipulation von Graphen.
Sie beweist sich dabei als besonders nützlich in Anwendungsgebieten wie der Charakterisierung von Expandergraphen, dem Graphenzeichnen oder dem spektralen Clustering~(\vgl{}~\cite{Shuman}).
Weiterhin hat die spektrale Graphentheorie \zB{} auch Anwendungsgebiete in der Chemie, bei der die Eigenwerte des Spektrums des Graphen mit der Stabilität von Molekülen assoziert werden (\vgl{}~\cite{Chung}).

Es zeigt sich, dass die Eigenwerte des Spektrums eines Graphen eng mit den Eigenschaften eines Graphen verwandt sind.
Als spektrale Graphentheorie versteht man damit insbesondere die Studie über die gemeinsamen Beziehungen dieser beiden Bereiche.
Dieses Kapitel gibt eine Einführung in die wichtigsten Definitionen und Intuitionen der spektralen Graphentheorie, die es uns schlussendlich erlauben, die spektrale Faltung auf Graphen zu formulieren.

\subsection{Eigenwerte und Eigenvektoren reell symmetrischer Matrizen}
\label{eigenwerte_symmetrischer_matrizen}

Das \emph{Eigenwertproblem} einer Matrix $\ma{M} \in \gls{R}^{N \times N}$ ist definiert als $\ma{M}\gls{eiv} = \gls{lambda}\gls{eiv}$, wobei $\gls{eiv} \in \gls{R}^{N}$, $\gls{eiv} \neq \mathbf{0}$ \emph{Eigenvektor} und $\gls{lambda} \in \gls{R}$ der entsprechende \emph{Eigenwert} zu \gls{eiv} genannt werden~\cite{linear}.
Ein Eigenvektor \gls{eiv} beschreibt damit einen Vektor, dessen Richtung durch die Abbildung $\ma{M}\gls{eiv}$ nicht verändert, sondern lediglich um den Faktor \gls{lambda} skaliert wird.
Zu einem Eigenwert $\gls{lambda}$ gibt es unendlich viele (skalierte) Eigenvektoren \gls{eiv}.
Wir definieren den Eigenvektor \gls{eiv} eines Eigenwertes \gls{lambda} daher eindeutig über die Bedingung $\left\|\gls{eiv}\right\|_2 = 1$.
Sei \ma{M} weiterhin symmetrisch, \dhe{} $\ma{M} = \ma{M}^{\top}$~\cite{linear}.
Dann gilt für zwei unterschiedliche Eigenvektoren $\gls{eiv}_1$ und $\gls{eiv}_2$, dass diese orthogonal zueinander stehen, \dhe{} $\gls{eiv}_1 \gls{ortho} \gls{eiv}_2$, und \ma{M} genau $N$ reelle Eigenwerte mit ${\left\{\gls{lambda}_n\right\}}_{n=1}^N$ hat~\cite{linear}.
Wir definieren demnach zu \ma{M} die orthogonale \emph{Eigenvektormatrix} $\gls{Eiv} \coloneqq \left[\gls{eiv}_1, \ldots, \gls{eiv}_n\right] \in \gls{R}^{N \times N}$, \dhe{} $\gls{Eiv}\gls{Eiv}^{\top}=\gls{Eiv}^{\top}\gls{Eiv}=\gls{I}$, und dessen Eigenwertdiagonalmatrix $\gls{Lambda} \coloneqq \gls{diag}\left({\left[\gls{lambda}_1, \ldots, \gls{lambda}_N\right]}^{\top}\right)$, \dhe{} $\gls{Lambda}_{ii} = \gls{lambda}_i$~\cite{Defferrard}.
Dann gilt $\ma{M}\gls{Eiv} = \gls{Eiv}\gls{Lambda}$ und insbesondere ist \ma{M} diagonalisierbar über~\cite{linear}
\begin{equation*}
  \ma{M} = \left(\ma{M}\gls{Eiv}\right)\gls{Eiv}^{\top} = \left(\gls{Eiv}\gls{Lambda}\right)\gls{Eiv}^{\top}.
\end{equation*}

Weiterhin gilt für die $k$-te Potenz von $\ma{M}$, $k \in \gls{N}$,~\cite{gcn}
\begin{equation}
  \ma{M}^k = {\left(\gls{Eiv}\gls{Lambda}\gls{Eiv}^{\top}\right)}^k = \gls{Eiv}\gls{Lambda}^k\gls{Eiv}^{\top}.
  \label{eq:matrix_potenz}
\end{equation}

aufgrund der Induktion ($k - 1 \rightarrow k$)
\begin{equation*}
  {\left(\gls{Eiv}\gls{Lambda}\gls{Eiv}^{\top}\right)}^k = {\left(\gls{Eiv}\gls{Lambda}\gls{Eiv}^{\top}\right)}^{k-1}\gls{Eiv}\gls{Lambda}\gls{Eiv}^{\top} = \gls{Eiv}\gls{Lambda}^{k-1}\gls{Eiv}^{\top}\gls{Eiv}\gls{Lambda}\gls{Eiv} = \gls{Eiv}\gls{Lambda}^k\gls{Eiv}^{\top}.
\end{equation*}

Falls \ma{M} weiterhin \emph{schwach diagonaldominant} ist, \dhe{}
\begin{equation}
  \sum_{\substack{j=1\\j \neq i}}^N \left|\ma{M}_{ij}\right| \leq \left|\ma{M}\right|_{ii},
  \label{eq:schwach_diagonaldominant}
\end{equation}
und weiterhin $\ma{M}_{ii} \geq 0$ für alle $i \in \left\{1, \ldots, N\right\}$, dann ist \ma{M} \emph{positiv semidefinit}, \dhe{} $\ve{x}^{\top}\ma{M}\ve{x} \geq 0$ für alle $\ve{x} \in \gls{R}^{N}$~\cite{linear}.
Eigenwerte symmetrischer positiv semidefiniter Matrizen $\lambda_i \in \gls{R+}$ sind positiv reell und es lässt sich folglich auf diesen eine Ordnung definieren mit $0 \leq \gls{lambda}_1 \leq \cdots \leq \gls{lambda}_N \coloneqq \gls{lambdamax}$~\cite{linear}.

\subsection{Laplace-Matrix}
\label{laplace_matrix}

Die Laplace-Matrix ist in der spektralen Graphentheorie eine Matrix, die die Beziehungen der Knoten und Kanten eines beliebigen Graphen \gls{G} in einer generalisierten und normalisierten Form beschreibt.
Viele der Eigenschaften von \gls{G} können durch die Eigenwerte ihrer Laplace-Matrix beschrieben werden, wohingegen dies \bspw{} für die Eigenwerte der Adjazenzmatrix \gls{A} von \gls{G} nur bedingt zutrifft und insbesondere nicht verallgemeinbar für beliebge Graphen ist~\cite{Chung}.
Dies ist vor allem dem Fakt geschuldet, dass die Eigenwerte der Laplace-Matrix konsistent sind mit den Eigenwerten des Laplace-Beltrami Operators $\nabla^2$ in der spektralen Geometrie~\cite{Chung}.
Die Laplace-Matrix ist damit ein geeignetes Mittel zur Betrachtung und Analyse eines Graphen.

Für einen schleifenlosen, ungerichteteten, gewichtet oder ungewichteten Graphen \gls{G} und dessen Adjazenzmatrix \gls{A} mit Gradmatrix \gls{D} ist die \emph{kombinatorische Laplace-Matrix} \gls{L} definiert als $\gls{L} \coloneqq \gls{D} - \gls{A}$~\cite{Chung}.
Die \emph{normalisierte Laplace-Matrix} \gls{Lnorm} ist definiert als $\gls{Lnorm} \coloneqq \gls{D}^{-1/2} \gls{L} \gls{D}^{-1/2}$ mit der Konvention, dass $\gls{D}^{-1/2}_{ii} = 0$ für isolierte Knoten $\gls{v}_i \in \gls{V}$ in \gls{G}, \dhe{} $\gls{D}_{ii} = 0$~\cite{Chung}.
Daraus ergibt sich die elementweise Definition
\begin{equation*}
  \gls{Lnorm}_{ij} \coloneqq \begin{cases}
  1, & \text{wenn }i = j,\\
    -\frac{\gls{w}\left(\gls{v}_i, \gls{v}_j\right)}{\sqrt{\gls{d}\left(\gls{v}_i\right)\gls{d}\left(\gls{v}_j\right)}}, & \text{wenn }\gls{v}_j \in \gls{Neighbor}\left(\gls{v}_i\right),\\
  0, & \text{sonst.}
\end{cases}
\end{equation*}
Für zusammenhängende Graphen kann \gls{Lnorm} vereinfacht werden zu~\cite{Chung}
\begin{equation}
  \gls{Lnorm} \coloneqq \gls{I} - \gls{D}^{-1/2} \gls{A} \gls{D}^{-1/2}.
  \label{eq:laplace_norm_connected}
\end{equation}
Jeder Eintrag auf der Diagonalen der normalisierten Laplace-Matrix ist folglich Eins.
\gls{Lnorm} ist damit normalisiert auf den (gewichteten) Grad zweier adjazenter Knoten $\gls{v}_i$ und $\gls{v}_j$.
Es ist anzumerken, dass \gls{L} und insbesondere \gls{Lnorm} symmetrisch sind, wohingegen eine Normalisierung der Form $\gls{D}^{-1}\gls{L}$ dies in der Regel nicht wäre~\cite{Reuter}.
\gls{L} und \gls{Lnorm} sind desweiteren keine ähnlichen Matrizen, insbesondere sind ihre Eigenvektoren verschieden.
Die Nutzung von \gls{L} oder \gls{Lnorm} ist damit abhängig von dem Problem, welches man betrachtet~\cite{Hammond}.
Wir schreiben \gls{Lboth} wenn die Wahl der Laplace-Matrix, ob \gls{L} oder \gls{Lnorm}, für die weitere Berechnung irrelevant ist.

\paragraph{Interpretation}
\label{laplace_interpretation}


\begin{figure}[t]
  \centering
  \subfigure[]{
    \begin{tikzpicture}
      \tikzstyle{node}=[circle,draw, minimum width=5pt, inner sep=0pt, fill=white]
      \tikzstyle{root}=[fill=black]

      \node[node, root] (root) at (0,0) {};
      \node[node, label={}] (a) at (-2, 0) {};
      \node[node, label={}] (b) at (2, 0) {};
      \node[node, label={}] (c) at (0, -2) {};
      \node[node, label={}] (d) at (0, 2) {};

      \path (root) edge node[above] {$1/\delta^2$} (a);
      \path (root) edge node[below] {$1/\delta^2$} (b);
      \path (root) edge node[left] {$1/\delta^2$} (c);
      \path (root) edge node[right] {$1/\delta^2$} (d);
    \end{tikzpicture}
  }
  \hspace{1cm}
  \subfigure[]{
    \begin{tikzpicture}
      \tikzstyle{node}=[circle,draw, minimum width=5pt, inner sep=0pt, fill=white]
      \tikzstyle{root}=[fill=black]

      \node[node, root] (root) at (0,0) {};
      \node[node, label={$\left(x-\delta, y\right)$}] (a) at (-2, 0) {};
      \node[node, label={$\left(x+\delta, y\right)$}] (b) at (2, 0) {};
      \node[node, label={[shift={(1.2, -0.22)}]$\left(x, y+\delta\right)$}] (c) at (0, -2) {};
      \node[node, label={[shift={(-1.2, -0.6)}]$\left(x, y-\delta\right)$}] (d) at (0, 2) {};

      \path (root) edge (a);
      \path (root) edge (b);
      \path (root) edge (c);
      \path (root) edge (d);
    \end{tikzpicture}
  }
  \caption[bla]{Testbilder}
  \label{fig:testbild}
\end{figure}


Sei $f \colon \gls{V} \to \gls{R}$ \bzw{} $\gls{f} \in \gls{R}^N$ mit $f\left(\gls{v}_i\right) = \gls{f}_i$ eine Funktion \bzw{} ein Signal auf den Knoten eines Graphen \gls{G}.
Dann kann für die kombinatorische Laplace-Matrix \gls{L} verifiziert werden, dass sie die Gleichung
\begin{equation*}
  {\left(\gls{L}\gls{f}\right)}_i = \sum_{\gls{v}_j \in \gls{Neighbor}\left(\gls{v}_i\right)} \gls{w}\left(\gls{v}_i, \gls{v}_j\right) \left(\gls{f}_i - \gls{f}_j\right)
\end{equation*}
erfüllt~\cite{Hammond}.
Sei $\gls{G}$ nun ein Graph, der aus einem (unendlichen) zweidimensionalen regulärem Gitter entstanden ist, \dhe{} jeder Knoten $\gls{v}_i$ besitzt genau vier achsenparallele rechtwinklige Nachbarn mit gleichen Kantengewichten $1/\delta^2$, wobei $\delta \in \gls{R}$ den Abstand der Knoten zueinander beschreibt.
Zur einfacheren Veranschaulichung benutzen wir dabei für die Signalstärke $\gls{f}_i$ eines Knoten $v_i$ an Position $\left(x, y\right)$ die Indexnotation $\gls{f}_{x,y}$.
Dann beschreibt
\begin{equation*}
  {\left(\gls{L}\gls{f}\right)}_{x,y} = \frac{4\gls{f}_{x,y} - \gls{f}_{x+1,y} - \gls{f}_{x-1,y} - \gls{f}_{x,y+1} - \gls{f}_{x,y-1}}{h^2}
\end{equation*}
die \emph{5-Punkte-Stern} Approximation $\nabla^2 f$ (bei umgekehrtem Vorzeichen) definiert auf den Punkten $\left\{\left(x,y\right), \left(x+\delta,y\right), \left(x-\delta,y\right), \left(x,y + \delta\right),\left(x,y-\delta\right)\right\}$~\cite{Hammond} (\vgl{} Abbildung~\ref{fig:5_punkte_stern}).
Ähnlich zu einem regulären Gitter lässt sich ein Graph \gls{G} auch über beliebig viele Abtastpunkte einer differenzierbaren Mannigfaltigkeit konstruieren.
Es zeigt sich, dass mit steigender Abtastdichte und geeigneter Wahl der Kantengewichte die normalisierte Laplace-Matrix \gls{Lnorm} zu dem kontinuierlichem Laplace-Beltrami Operator $\nabla^2$ konvergiert~\cite{Hammond}.
Damit kann $\gls{Lnorm}$ als die diskrete Analogie des $\nabla^2$ Operators auf Graphen verstanden werden.
Der Laplace-Beltrami Operator $\nabla^2 f\left(p\right)$ misst dabei, in wie weit sich eine Funktion $f$ an einem Punkt $p$ von dem Durchschnitt aller Funktionspunkte um einen kleinen Bereich um $p$ unterscheidet.
Die Laplace-Matrix operiert dabei völlig analog, in dem sie misst, wie sehr sich eine (diskrete) Funktion um einen Knoten im Vergleich zu seinen Nachbarknoten unterscheidet.

Die Eigenwerte und Eigenvektoren von \gls{Lboth} helfen uns beim Verständnis der linearen Transformation einer Funktion \gls{f} (mehrfach) angewendet auf \gls{Lboth}.
Wir können dafür \gls{f} als Linearkombination der Eigenbasis $\gls{f} = \sum_{n=1}^N a_n \gls{eiv}_n$, $a_n \in \gls{R}$ schreiben und erhalten
\begin{equation*}
  \gls{Lboth}^k \gls{f} = \sum_{n=1}^N a_n \gls{Lboth}^k \gls{eiv}_n = \sum_{n=1}^N a_n \gls{lambda}_n^k \gls{eiv}_n.
\end{equation*}
Somit können Eigenschaften von \gls{Lboth} und damit des Graphen selber durch dessen Eigenwerte und Eigenvektoren beschrieben werden.

\paragraph{Eigenschaften}
\label{laplace_eigenschaften}

$\gls{Lboth} \in \gls{R}^{N \times N}$ ist eine reell symmetrische, positiv semidefinite Matrix~\cite{Chung}.
Folglich besitzt \gls{Lboth} nach Kapitel~\ref{eigenwerte_symmetrischer_matrizen} genau $N$ positiv reelle Eigenwerte ${\left\{\gls{lambda}_n\right\}}_{n=1}^N$ mit Ordnung $0 \leq \gls{lambda}_1 \leq \cdots \leq \gls{lambda}_N$ und $N$ korrespondierenden orthogonalen Eigenvektoren ${\left\{\gls{eiv}_n\right\}}_{n=1}^N$.

Die kombinatorische Laplace-Matrix $\gls{L}$ ist nach~\eqref{eq:schwach_diagonaldominant} weiterhin schwach diagonaldominant.
Insbesondere summiert sich jede Zeilen- und Spaltensumme von \gls{L} zu Null auf, \dhe{} $\sum_{j=1}^N \gls{L}_{ij} = \sum_{j=1}^N \gls{L}_{ji} = 0$.
Daraus folgt unmittelbar, dass $\gls{lambda}_1 = 0$, da $\gls{eiv}_1 = 1/\sqrt{N}{\left[1, \ldots, 1\right]}^{\top} \in \gls{R}^N$ Eigenvektor von \gls{L} mit $\gls{L}\gls{eiv}_1 = \ve{0}$~\cite{Shuman}.
\gls{Lnorm} hingegen ist nicht zwingend schwach diagonaldominant.
Es lässt sich jedoch zeigen, dass auch für \gls{Lnorm} gilt, dass $\gls{lambda}_1 = 0$~\cite{Chung}.

Eine der interessantesten Eigenschaften eines Graphen ist dessen Konnektivität.
Die Laplace-Matrix \gls{Lboth} \bzw{} deren Eigenwerte stellen ein geeignetes Mittel zur Untersuchung dieser Eigenschaft dar.
So gilt \bspw{} für einen zusammenhängenden Graphen \gls{G}, dass $\gls{lambda}_2 > 0$.
Falls $\gls{lambda}_i = 0$ und $\gls{lambda}_{i+1} \neq 0$, dann besitzt $\gls{G}$ genau $i$ zusammenhängende Komponenten~\cite{Chung}.
Damit ist die Anzahl der Null-Eigenwerte äquivalent zu der Anzahl an Komponenten, die ein Graph besitzt.
Für \gls{Lnorm} lässt sich weiterhin zeigen, dass $\gls{lambdamax} \leq 2$ eine obere Schranke ihrer Eigenwerte ist~\cite{Chung}.

Aus der Laplace-Matrix können ebenso Rückschlüsse über die kürzeste Pfaddistanz zweier Knoten gewonnen werden.
So gilt für $\gls{Lboth}^{k}$ mit $k \in \gls{N}$, dass $\gls{Lboth}^k_{ij} = 0$ genau dann, wenn $\gls{s}\left(v_i, v_j\right) > k$~\cite{Hammond}.
Damit beschreibt $\gls{Lboth}^k_i$ bildlich gesprochen die Menge an Knoten, die maximal $k$ Kanten von $\gls{v}_i$ entfernt liegen.
