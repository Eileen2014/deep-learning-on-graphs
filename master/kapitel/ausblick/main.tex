\chapter{Ausblick}
\label{ausblick}

Die vorgestellten und entwickelten Ansätze zum Lernen auf Graphen im zweidimensional euklidischen Raum wurden in dieser Arbeit über einer Klassizierung auf Graphrepräsentationen von Bildern evaluiert.
Sie lassen sich jedoch auch mühelos auf andere Probleme wie \zB{} der Objekterkennung oder der Segmentierung anwenden.
Dabei sind ebenso völlig andere als in dieser Arbeit präsentierten Anwendungsgebiete denkbar, denn die einzige Einschränkung der Graphen für die Benutzung der entwickelten Faltungsoperatoren ist die eindeutige oder zumindest relative Positionierung seiner Knoten zueinander im Raum.
So lassen sich solche Graphen \bspw{} über Karten oder Straßennetze gewinnen, deren Eingabe in die vorgestellten Netzarchitekturen vielversprechende Möglichkeiten offenbaren.

Zum Abschluss dieser Arbeit wird ein Ausblick zu denkbaren Erweiterungen vorgestellt, für dessen Erforschung an dieser Stelle der Platz fehlt.

\paragraph{Erweiterung auf den dreidimensionalen Raum}
\label{dredimensionale_erweiterung}

Alle Faltungsansätze in dieser Arbeit wurden lediglich auf Graphen im zweidimensionalen Raum betrachtet.
Dabei ist es aber ebenso vorstellbar, diese Ansätze auf den dreidimensionalen Raum auszuweiten.
In diesem Fall besitzt ein Graph $\gls{G} = \left(\gls{V}, \gls{E}, \gls{p}\right)$ eine Positionsfunktion $\gls{p} \colon \gls{V} \to \gls{R}^3$, die den Knoten jeweils drei Koordinaten zuordnet.
Die implizit gegebene Winkelfunktion seiner Kanten $\gls{winkel} \colon \gls{V} \times \gls{V} \to {\left[0, 2\pi\right]}^2$ bildet damit folglich auf zwei Winkel ab.
Insbesondere der spektrale Faltungsoperator, der in Kapitel~\ref{bspline} mit Hilfe einer polynomiellen Approximation über B-Spline-Kurven erweitert wurde, erlernt nun eine Repräsentation einer geschlossenen Fläche anstatt einer Kurve, die mittels der \emph{Tensorproduktkonstruktion} über zwei Basisfunktionen konstruiert werden kann~\cite{gm}.
Eine Implementierung \bzw{} Erweiterung dieser Ansätze auf den dreidimensionalen Raum erlaubt damit insbesondere das Lernen auf Polygonnetzen und folglich auch auf dreidimensionalen Objekten.

\paragraph{Entfernung irrelevanter Knoten}
\label{entfernung_irrelevanter_knoten}

In den Graphrepräsentationen von Bildern können bei der Benutzung der Quickshift-Segmentierung Knoten vereinzelnd einen extrem hohen Knotengrade aufweisen (\vgl{} \zB{} Tabelle~\ref{tab:mnist}).
Das ist insbesondere dann der Fall, wenn Knoten eine große einheitliche Fläche des Bildes abdecken (\zB{} einen einfarbigen Hintergrund) und folglich an dessen Segmenträndern viele benachbarte Pixel anliegen.
Eine Faltung auf diesen Knoten besitzt so gut wie keine lokale Aussage, sodass dessen Nutzen für ein neuronales Netzes fraglich oder gar hinderlich ist.
Es erscheint sinnvoll, diese Knoten in einem weiteren Vorverarbeitungsschritt über einer zuvor definierten Strategie zu entfernen.
Abbildung~\ref{fig:knotenentfernung} illustriert eine mögliche irrelevante Knotenentfernung anhand eines Bildes aus dem \gls{MNIST}-Datensatz.
Es lassen sich folglich zwischen datensatzspezifischen und -unspezifischen Strategien unterscheiden.
Auf dem \gls{MNIST} Datensatz können \zB{} alle Knoten entfernt werden, die einen schwarzen Farbwert oder eine (relativ) große Fläche aufweisen.
Eine allgemeinere Vorgehensweise ist die Knotenentfernung basierend auf einem (relativen) Grenzwert seiner Knotengrade.
Bei der bloßen Entfernung von Knoten aus einem Graphen laufen wir aber Gefahr, dass dieser nun möglicherweise isolierte Knoten \bzw{} mehrere Zusammenhangskomponenten besitzt.
Insbesondere die Poolingoperation sowie die Faltung basierend auf den \glspl{GCN} erfordern jedoch zusammenhängende Graphen.
Ein Test mit Hilfe des zweiten Eigenwerts der Laplace-Matrix zeigt, dass die Knotenentfernung basierend auf ihren Farbwerten für Graphen, die über Quickshift aus dem \gls{MNIST}-Datensatz generiert wurden, zu $99.9\%$ einen zusammenhängenden Graphen erhält, wohingegen für eine Knotenentfernung auf Basis ihrer Knotengrade dies nur noch für $45.4\%$ der Graphen der Fall ist.
Es lassen sich jedoch die beiden nächsten zueinander liegenden Knoten unterschiedlicher Zusammenhangskomponenten über eine Kante verbinden, bis der Graph wieder zusammenhängend ist.

\paragraph{Spatial-Pyramid-Pooling}
\label{spatial_pyramid_pooling}

\paragraph{Effiziente GPU-Implementierung}
\label{gpu_implementierung}

Erweiterung auf den dreidimensionalen Raum

Entfernung irrelevanter Kanten

Spatial Pyramid Pooling als Alternative zur durchschnittsbildung


spannende mäglichkeit verspricht

Im folgenden

Meshes 3d





\paragraph{Attention-Algorithmus}
\label{attention_algorithmus}
