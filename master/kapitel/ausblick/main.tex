\chapter{Ausblick}
\label{ausblick}

\paragraph{Weitere Anwendungsgebiete}
\label{weitere_anwendungsgebiete}

\paragraph{Entfernung irrelevanter Knoten}
\label{entfernung_irrelevanter_knoten}

nicht nur bei MNIST
auch bei pascal voc slic? (\zB{} Regelmäßigkeiten erkennen)

Problem kanten bzw knoten mit sehr vielen kanten enthalten wenig information und sind meist hintergrundsknoten, die eine grosse gleichfarvige fläche bilden (das ist natürlich nicht immer der fall)
Wir können diese knoten über einen treshhold entfernen (zb 20)
Laufen jedoch gefahr isolierte knoten zu schaffen bzw einen nicht verbundenen graphen (zeige beispiel)
 Eine weitere idee ist für jeden knoten einen threshold an anzahl an kanten zusetzen und die längsten kanten zu entfernen. Dann bleibt aber der unnütze hintergrundknoten erhalten.

Im falle von mnist kann mn hintergrundsknoten leicht über farbe ermitteln.
In anderen interessanteren datensätzen ist dies nicht so einfach.

Gibt es weitere ideen?
Kann man isolierte graphen leicht ermitteln? Was tun in so einem fall?

Testen über Eigenvektoren ob Graph zusammenhängend ist

\paragraph{Spatial-Pyramid-Pooling}
\label{spatial_pyramid_pooling}

\paragraph{Attention-Algorithmus}
\label{attention_algorithmus}
