\section{Mathematische Notationen}
\label{mathematische_notationen}

Eine \emph{ungeordnete Menge} $\mathcal{A} \coloneqq \left\{a_1, \ldots, a_N\right\} = {\left\{a\right\}}_{n=1}^N$, $N \in \gls{N}$, beschreibt eine Gruppierung von $N$ Elementen $a_n$ mit $1 \leq n \leq N$.
Die Menge $\mathcal{A}$ besitzt damit die \emph{Kardinalität} $\left|\mathcal{A}\right| = N$.
Eine Menge $\mathcal{A}$ heißt \emph{geordnet}, wenn auf dessen Elementen eine reflexive, transitive und antisymmetrische Ordnung $\mathcal{A} \times \mathcal{A}$ definiert ist~\cite{linear}.
Eine geordnete Menge $\mathcal{A}$ mit $N$ Elementen wird über $\mathcal{A} \coloneqq \left(a_1, \ldots, a_N\right)$ gekennzeichnet, wobei die Reihenfolge der Elemente $\left(a_1, \ldots, a_N\right)$ dessen Ordnung beschreiben.

Sei $\ve{v} \in \gls{R}^N$, $N \in \gls{N}$, mit $\ve{v} = {\left[v_1, \ldots, v_n \right]}^{\top}$ ein \emph{Vektor} mit $N$ reellen Elementen ${\left\{v_n\right\}}_{n=1}^N$, $v_n \in \gls{R}$, wobei $\ve{v}_n = v_n$ das $n$-te Element von $\ve{v}$ referenziert.
Zu zwei Vektoren $\ve{v}, \ve{w} \in \gls{R}^N$ ist das \emph{Skalarprodukt} $\left\langle \ve{v}, \ve{w} \right\rangle$ definiert als $\left\langle \ve{v}, \ve{w}\right\rangle \coloneqq \sum_{n=1}^N \ve{v}_n \ve{w}_n$~\cite{linear}.
$\ve{v}$ und $\ve{w}$ stehen \emph{orthogonal} zueinander, \dhe{} $\ve{v} \gls{ortho} \ve{w}$, \gdw{} $\left\langle \ve{v}, \ve{w} \right\rangle = 0$~\cite{linear}.

Eine \emph{Matrix} $\ma{M} \in \gls{R}^{N \times M}$, $N, M \in \gls{N}$, erweitert einen Vektor um $M$ Spalten.
Der Wert $\ma{M}_{nm} \in \gls{R}$ beschreibt damit das Element in der $n$-ten Zeile und $m$-ten Spalte von \ma{M}.
Die \emph{transponierte Matrix} $\ma{M}^{\top} \in \gls{R}^{M \times N}$ ist eine Matrix, die aus $\ma{M} \in \gls{R}^{N \times N}$ durch Vertauschen der Zeilen und Spalten entsteht, \dhe{} $\ma{M}^{\top}_{mn} = \ma{M}_{nm}$~\cite{linear}.
Zu einem Vektor $\ve{v} \in \gls{R}^{N}$ lässt sich weiterhin dessen korrespondiere quadratische \emph{Diagonalmatrix} $\gls{diag}\left(\ve{v}\right) \in \gls{R}^{N \times N}$
über
\begin{equation*}
  \gls{diag}\left(\ve{v}\right) \coloneqq \begin{bmatrix}
  \ve{v}_1 & & \ve{0}\\
  & \ddots & \\
  \ve{0} & & \ve{v}_N
  \end{bmatrix}
\end{equation*}
\bzw{} ${\gls{diag}\left(\ve{v}\right)}_{nn} \coloneqq \ve{v}_n$ definieren~\cite{Defferrard}.

\paragraph{Dünnbesetzte Matrizen}

Eine \emph{dünnbesetzte} oder \emph{schwachbesetzte Matrix} ist eine Matrix, bei der so viele Einträge aus Nullen bestehen, dass sich statt der üblichen Speicherung einer Matrix als zweidimensionales Feld speichereffizientiere Datenstrukturen ergeben.
In der Regel gilt eine Matrix $\ma{M}\in \gls{R}^{N \times N}$ als dünnbesetzt, wenn diese nicht aus mehr als $N$ oder $N \log N$ Einträgen ungleich Null besteht.
Neben dem Speichergewinn lassen sich viele Operationen auf Matrizen ebenso berechnungseffizienter implementieren~\cite{Saad}.
So muss \zB{} zur Bestimmung des größten Elements einer Matrix nicht die komplette Matrix, sondern lediglich deren explizit eingetragene Werte betrachtet werden.
Es gibt jedoch auch Operationen, die nicht auf dünnbesetzten Matritzen definiert sind, sodass diese vorher in eine \emph{dichte} Matrix überführt werden müssen (\vgl{}~\cite{Saad}).
Im Laufe dieser Arbeit haben wir es oft mit dünnbesetzten Matrizen zu tun.
So wird \zB{} eine Diagonalmatrix \emph{immer} als eine dünnbesetzte Matrix implementiert.

\paragraph{Tensoren}

Ein \emph{Tensor} $\ma{T} \in \gls{R}^{N_1 \times \cdots \times N_R}$, $N_1, \ldots, N_R \in \gls{N}$, ist ein mathematisches Objekt, welches das Konzept der Vektoren und Matrizen auf beliebig viele Dimensionen erweitert.
Die Anzahl der Dimensionen $R$ eines Tensors wird auch \emph{Rang} genannt~\cite{linear}.
Vektoren können damit insbesondere als Tensor mit Rang $1$ sowie Matrizen als Tensor mit Rang $2$ verstanden werden.
Ein Tensor mit Rang $0$ beschreibt ein Skalar.
Aus einem Tensor $\ma{T} \in \gls{R}^{N_1 \times \cdots \times N_R}$ können über die Indexnotation Tensoren mit geringerer Dimension gefiltert werden.
So beschreibt $\ma{T}_{n_1 \cdots n_r}$ einen Tensor mit Rang $0$ \bzw{} das Element des Tensors an Position $n_1, \ldots, n_r$.
Analog beschreibt \zB{} $\ma{T}_{n_1} \in \gls{R}^{N_2 \times \cdots \times N_R}$ einen Tensor mit Rang $R-1$, bei dem die erste Dimension über $n_1$ festgehalten wird.

\paragraph{Bilder}

Ein \emph{Bild} kann folglich durch einen dreidimensionalen Tensor $\gls{B} \in \gls{R}^{H \times W \times C}$ repräsentiert werden, wobei $H, W \in \gls{N}$ die Höhe \bzw{} Breite des Bildes angeben und $C \in \left\{1, 3\right\}$ die Anzahl der Farbkanäle des Bildes beschreibt, \dhe{} ein Graubild mit nur einem Kanal oder ein Farbbild mit drei Kanälen (\zB{} über das RGB- oder Lab-Farbmodell).
Ein \emph{Pixel} eines Bildes \gls{B} an der Position $\left(x, y\right)$ mit $1 \leq x \leq W, 1 \leq y \leq H$ kann folglich über $\gls{B}_{yx} \in \gls{R}^C$ angesprochen werden.
