\section{Mathematische Notationen}
\label{mathematische_notationen}

\paragraph{Vektoren}
skalarprodukt

\paragraph{Matrizen}
Diagonalmatrix

\paragraph{Dünnbesetzte Matrizen}

Eine \emph{dünnbesetzte} oder \emph{schwachbesetzte Matrix} ist eine Matrix, bei der so viele Einträge aus Nullen bestehen, dass sich statt der üblichen Speicherung einer Matrix als zweidimensionales Feld speichereffizientiere Datenstrukturen ergeben.
In der Regel gilt eine Matrix $\ma{M}\in \gls{R}^{N \times N}$ als dünnbesetzt, wenn diese nicht aus mehr als $N$ oder $N \log N$ Einträgen ungleich Null besteht.
Neben dem Speichergewinn lassen sich viele Operationen auf Matrizen ebenso berechnungseffizienter implementieren~\cite{Saad}.
So muss \zB{} zur Bestimmung des größten Elements einer Matrix nicht die komplette Matrix, sondern lediglich deren explizit eingetragene Werte betrachtet werden.
Es gibt jedoch auch Operationen, die nicht auf dünnbesetzten Matritzen definiert sind, sodass diese vorher in eine \emph{dichte} Matrix überführt werden müssen (\vgl{}~\cite{Saad}).
Im Laufe dieser Arbeit haben wir es oft mit dünnbesetzten Matrizen zu tun.
So wird \zB{} eine Diagonalmatrix \emph{immer} als eine dünnbesetzte Matrix implementiert.

\paragraph{Tensoren}
Tensor, was bedeutet z.B. $\gls{W}_i$

\paragraph{Bilder}

Ein \emph{Bild} kann folglich durch einen dreidimensionalen Tensor $\gls{B} \in \gls{R}^{H \times W \times C}$ repräsentiert werden, wobei $H, W \in \gls{N}$ die Höhe \bzw{} Breite des Bildes angeben und $C \in \left\{1, 3\right\}$ die Anzahl der Farbkanäle des Bildes beschreibt, \dhe{} ein Graubild mit nur einem Kanal oder ein Farbbild mit drei Kanälen (\zB{} über das RGB- oder Lab-Farbmodell).
Ein \emph{Pixel} eines Bildes \gls{B} an der Position $\left(x, y\right)$ mit $1 \leq x \leq W, 1 \leq y \leq H$ kann folglich über $\gls{B}_{yx} \in \gls{R}^C$ angesprochen werden.

\paragraph{Mengen}

Eine ungeordnete Menge $\mathcal{A} = {\left\{a\right\}}_{n=1}^N \coloneqq \left\{a_1, \ldots, a_N\right\}$ mit $N$ Elementen, $a_n \in \mathcal{A}$.
Geordnete Menge $\left(\right)$
Eine geordnete Menge kann ebenso als Vektor, Matrix \bzw{} Tensor je nach Dimensionalität der Daten verstanden werden.
