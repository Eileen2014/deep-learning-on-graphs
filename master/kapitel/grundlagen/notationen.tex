\section{Mathematische Notationen}
\label{mathematische_notationen}

% Tensor, was bedeutet z.B. $\gls{W}_i$

% Diagonalmatrix
% skalarprodukt

\paragraph{Bilder}

Ein \emph{Bild} kann folglich durch einen dreidimensionalen Tensor $\gls{B} \in {\left[0, 1\right]}^{H \times W \times C}$ repräsentiert werden, wobei $H, W \in \gls{N}$ die Höhe \bzw{} Breite des Bildes angeben und $C \in \left\{1, 3\right\}$ die Anzahl der Farbkanäle des Bildes beschreibt, \dhe{} ein Graubild mit nur einem Kanal oder ein Farbbild mit drei Kanälen (\zB{} über das RGB-Farbmodell).
Die Werte der einzelnen Farbkanäle des Bildes werden dabei auf $\left[0, 1\right]$ skaliert.

\paragraph{Dünnbesetzte Matrizen}

Eine \emph{dünnbesetzte} oder \emph{schwachbesetzte Matrix} ist eine Matrix, bei der so viele Einträge aus Nullen bestehen, dass diese nicht explizit mitgespeichert werden müssen.
In der Regel gilt eine Matrix $\ma{M}\in \gls{R}^{N \times N}$ als dünnbesetzt, wenn diese nicht aus mehr als $N$ oder $N \log N$ Einträgen ungleich Null besteht.
Damit ergeben sich Möglichkeiten hinsichtlich Operationen und Speicherung, schneller zu sein, da nicht alle Werte der Matrix betrachtet werden müssen.
Es gibt jedoch auch Operationen, die nicht auf dünnbesetzten Matritzen definiert sind, sodass diese vorher in eine \emph{dichte} Matrix überführt werden müssen (\vgl{}~\cite{Saad}).
