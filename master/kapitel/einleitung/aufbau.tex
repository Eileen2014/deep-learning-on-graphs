\section{Aufbau der Arbeit}
\label{aufbau_der_arbeit}

Die vorliegende Arbeit gliedert sich, neben den in Kapitel~\ref{grundlagen} vorgestellten Grundlagen, die für das weitere Verständnis der Arbeit von Nöten sind, in vier Bereiche.

Das Kapitel~\ref{graphrepraesentationen_von_bildern} widmet sich der Gewinnung einer Graphrepräsentation aus einem Bild.
Dabei werden insbesondere zwei Verfahren zur Generierung eines Graphen aus einem Bild vorgestellt — die herkömmliche Gitterrepräsentation eines Bildes (Kapitel~\ref{gitter}) sowie die Gewinnung eines Graphen aus einer vorberechneten Superpixelrepräsentation (Kapitel~\ref{superpixel}).

Kapitel~\ref{raeumliches_lernen} und~\ref{spektrales_lernen} erläutern die beiden unterschiedlichen Ansätze des graphbasierten Lernens in neuronalen Netzen — das räumliche und das spektrale Lernen — getrennt voneinander.
Beide Verfahren umfassen den aktuellen Stand der wissenschaftlichen Forschung auf diesem Gebiet.
Zu jedem Ansatz finden sich in den Unterkapiteln~\ref{raeumliche_erweiterung} \bzw{}~\ref{gcn_erweiterung} entwickelte Erweiterungen \bzgl{} der beschriebenen Problemstellung.

Kapitel~\ref{evaluation} evaluiert die vorgestellten sowie entwickelten Ansätze anhand der gegebenen Problemstellung im Vergleich zueinander sowie im Vergleich zu klassischen Lösungen auf diesem Gebiet.
Dazu wird zunächst in Unterkapitel~\ref{versuchsaufbau} der Versuchsaufbau erläutert und dessen Ergebnisse in den Unterkapiteln~\ref{ergebnisse} und~\ref{laufzeitanalyse} präsentiert.
In Kapitel~\ref{ausblick} folgt ein abschließender Ausblick zu möglichen weiterführenden Arbeiten.
