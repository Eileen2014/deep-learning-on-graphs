\section{Umwandlung in eine Graph-Repräsentation}

\begin{itemize}
  \item jeder Superpixel bildet einen Knoten im Graphen
  \item es existiert eine Kante zwischen den Knoten, wenn die entsprechenden Superpixel benachbart sind oder die Distanz zwischen Superpixeln unter eine Schranke $\epsilon$ fällt (müsste aber durch die Distanz der Kanten abgedenkt sein im CNN auf Graphen)
  \item \underline{Knotenattribute:}
  \begin{itemize}
    \item Farbe (RGB)
    \item Schwerpunkt/Position
    \item Größe, d.h.\ Anzahl Pixel (prozentual?)
    \item Ausdehnunng/Form $\Rightarrow$ z.B. über vereinfachten Polygonzug
    \item minimales gedrehtes Hüllrechteck
  \end{itemize}
  \item \underline{Kantenattribute:}
  \begin{itemize}
    \item Distanz zu den Schwerpunkten der Superpixel
  \end{itemize}
\end{itemize}
