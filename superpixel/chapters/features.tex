\section{Features}

\subsection{Momente}

Momente sind in der Bildverarbeitung bestimmte gewichtete Milttelwerte aus den Helligkeitswerten der einzelnen Pixel eines Bildes.
Sie werden gewoehnlich so gewaehlt, dass sie gewuenschte Eigenschaften des Bildes widerspiegeln oder gewisse geometrische Interpretationen besitzen.
Momente sind hilfreich, um einzelne Objekkte in einem \textbf{segmentierten} Bild zu beschreiben.

\subsubsection{Nicht-zentrierte Momente}

\begin{equation}
  M_{ij} = \sum_x \sum_y x^i y^j g(x, y)
\end{equation}

\begin{itemize}
  \item \textbf{Flaeche:} $M_{00}$
  \item \textbf{Schwerpunkt:} $\frac{M_{10}}{M_{00}}$ und $\frac{M_{01}}{M_{00}}$
\end{itemize}

\subsubsection{Zentrale Momente}

Zentrale Momente sind invariant bezueglich Translationen.

\begin{equation}
  \mu_{ij} = \sum_x \sum_y (x - \overline{x})^i (y - \overline{y})^j g(x, y)
\end{equation}

wobei $\overline{x}$, $\overline{y}$ Schwerpunkt.

\begin{itemize}
  \item $\mu_{00} = M_{00}$
\end{itemize}

Informationen ueber die Ausrichtung des Bildes koennen gewonnen werden, indem man zuerst die drei zentralen Momente zweiten Grades verwendet, um eine Kovarianzmatrix zu berechnen

\begin{equation}
  \text{cov}[I(x,y)] = \begin{pmatrix}
    \frac{\mu_{20}}{\mu_{00}} & \frac{\mu_{11}}{\mu_{00}}\\
    \frac{\mu_{11}}{\mu_{00}} & \frac{\mu_{02}}{\mu_{00}}\\
  \end{pmatrix} = \begin{pmatrix}
    \mu'_{20} & \mu'_{11}\\
    \mu'_{11} & \mu'_{02}\\
  \end{pmatrix}
\end{equation}

Die Eigenvektoren dieser Matrix entsprechen der grossen und kleinen Halbachse der Helligkeitswerte.
Die Eigenwerte der Kovarianzmatrix sind

\begin{equation}
  \lambda_i = \frac{\mu'_{20} + \mu'_{02}}{2} \pm \frac{\sqrt{4 \mu'^2_{11} + (\mu'_{20} - \mu'_{02})^2}}{2}
\end{equation}

Die \emph{Exzentrizitaet} (engl. \emph{Eccentricity}) des Bildes ist $\sqrt{1 - \frac{\lambda_2}{\lambda_1}}$.
Sie ist ein Mass fuer das Groessenverhaeltnis der beiden Hauptachsen.
Bei runden Objekten ist sie nahe an $0$, bei laenglichen Objekten nahe an $1$.

\subsubsection{Skalierungsinvariante Momente}

Es koennen Momente $\eta_{ij}$ mit $i + j \geq 2$ konstruiert werden, die invariant bezueglich Skalierung und Translation sind, indem man das entsprechende zentrale Moment durch das entsprechend skalierte Moment vom Grad $0$ teilt.

\begin{equation}
  \eta_{ij} = \frac{\mu_{ij}}{\mu_{00}^{1 + \frac{i + j}{2}}}
\end{equation}

\subsubsection{Rotationsinvariante Momente}

Es ist weiterhin moeglich, Momente zu konstruieren, die zusaetzlich invariant bezueglich einer Bildrotation sind.
Haeufig benutzt wird die \emph{Hu}-Menge invarianter Momente.

\begin{itemize}
  \item $I_1 = \eta_{20} + \eta_{02}$: Traegheitsmoment um den Schwerpunkt des Bildes, wenn die Helligkeitswerte der Pixel als physikalische Dichte interpretiert werden
\end{itemize}

\subsection{Polygon-Features}

\begin{itemize}
  \item \textbf{Area:} Anzahl an Pixeln im Segment
  \item \textbf{Bounding Box Area:} Flaeche der vertikalen/horizontalen gelegenen Bounding Box um das Segment
  \item \textbf{Convex Area:} Anzahl an Pixeln der konvexen Huelle des Segments
  \item \textbf{Local Centroid:} Zentrum/Schwerpunkt des Segments relativ zur Bounding Box
  \item \textbf{Eccentricity:} Groessenverhaeltnis der beiden Hauptachsen zueinander $\in [0, 1]$
  \item \textbf{Major Axis Length:} $4 \sqrt{\lambda_1}$
  \item \textbf{Minor Axis Length:} $4 \sqrt{\lambda_2}$
\end{itemize}

Daraus koennen weitere Features ermittelt werden:

\begin{itemize}
  \item \textbf{Extent:} $\frac{\text{Area}}{\text{Bounding Box Area}}$
  \item \textbf{Solidity:} $\frac{\text{Area}}{\text{Convex Area}}$
\end{itemize}
