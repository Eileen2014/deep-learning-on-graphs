\newpage

\section{Features}

\subsection{Momente}

Momente sind in der Bildverarbeitung bestimmte gewichtete Milttelwerte aus den Helligkeitswerten der einzelnen Pixel eines Bildes.
Sie werden gewoehnlich so gewaehlt, dass sie gewuenschte Eigenschaften des Bildes widerspiegeln oder gewisse geometrische Interpretationen besitzen.
Momente sind hilfreich, um einzelne Objekkte in einem \textbf{segmentierten} Bild zu beschreiben.

Momente koennen je nach Wahl von $g$ entweder auf einem Helligkeitsbild oder auf einer Segmentierungsmaske agieren und haben unterschiedliche Bedeutung (gewichtet/ungewichtet).

\subsubsection{Nicht-zentrierte Momente}

\begin{equation}
  M_{ij} = \sum_x \sum_y x^i y^j g(x, y)
\end{equation}

\begin{itemize}
  \item \textbf{Flaeche:} $M_{00}$
  \item \textbf{Schwerpunkt:} $\frac{M_{10}}{M_{00}}$ und $\frac{M_{01}}{M_{00}}$
\end{itemize}

\subsubsection{Zentrale Momente}

Zentrale Momente sind invariant bezueglich Translationen.

\begin{equation}
  \mu_{ij} = \sum_x \sum_y (x - \overline{x})^i (y - \overline{y})^j g(x, y)
\end{equation}

wobei $\overline{x}$, $\overline{y}$ Schwerpunkt.

\begin{itemize}
  \item $\mu_{00} = M_{00}$
\end{itemize}

Informationen ueber die Ausrichtung des Bildes koennen gewonnen werden, indem man zuerst die drei zentralen Momente zweiten Grades verwendet, um eine Kovarianzmatrix zu berechnen

\begin{equation}
  \text{cov}[I(x,y)] = \begin{pmatrix}
    \frac{\mu_{20}}{\mu_{00}} & \frac{\mu_{11}}{\mu_{00}}\\
    \frac{\mu_{11}}{\mu_{00}} & \frac{\mu_{02}}{\mu_{00}}\\
  \end{pmatrix} = \begin{pmatrix}
    \mu'_{20} & \mu'_{11}\\
    \mu'_{11} & \mu'_{02}\\
  \end{pmatrix}
\end{equation}

Die Eigenvektoren dieser Matrix entsprechen der grossen und kleinen Halbachse der Helligkeitswerte.
Die Eigenwerte der Kovarianzmatrix sind

\begin{equation}
  \lambda_i = \frac{\mu'_{20} + \mu'_{02}}{2} \pm \frac{\sqrt{4 \mu'^2_{11} + (\mu'_{20} - \mu'_{02})^2}}{2}
\end{equation}

Die \emph{Exzentrizitaet} (engl. \emph{Eccentricity}) des Bildes ist $\sqrt{1 - \frac{\lambda_2}{\lambda_1}}$.
Sie ist ein Mass fuer das Groessenverhaeltnis der beiden Hauptachsen.
Bei runden Objekten ist sie nahe an $0$, bei laenglichen Objekten nahe an $1$.

\subsubsection{Skalierungsinvariante Momente (normalisiert)}

Es koennen Momente $\eta_{ij}$ mit $i + j \geq 2$ konstruiert werden, die invariant bezueglich Skalierung und Translation sind, indem man das entsprechende zentrale Moment durch das entsprechend skalierte Moment vom Grad $0$ teilt.

\begin{equation}
  \eta_{ij} = \frac{\mu_{ij}}{\mu_{00}^{1 + \frac{i + j}{2}}}
\end{equation}

\subsubsection{Rotationsinvariante Momente}

Es ist weiterhin moeglich, Momente zu konstruieren, die zusaetzlich invariant bezueglich einer Bildrotation sind.
Haeufig benutzt wird die \emph{Hu}-Menge invarianter Momente.

\begin{itemize}
  \item $I_1 = \eta_{20} + \eta_{02}$: Traegheitsmoment um den Schwerpunkt des Bildes, wenn die Helligkeitswerte der Pixel als physikalische Dichte interpretiert werden
\end{itemize}

\subsection{Polygon-Features}

\begin{itemize}
  \item \textbf{Area:} Anzahl an Pixeln im Segment
  \item \textbf{Bounding Box Area:} Flaeche der vertikal/horizontal gelegenen Bounding Box um das Segment
  \item \textbf{Bounding Box Height/Width:} Hoehe bzw. Breite der vertikal/horizontal gelegenen Bounding Box
  \item \textbf{Convex Area:} Anzahl an Pixeln der konvexen Huelle des Segments
  \item \textbf{Local Centroid:} Zentrum/Schwerpunkt des Segments relativ zur Bounding Box
  \item \textbf{Eccentricity:} Groessenverhaeltnis der beiden Hauptachsen einer Ellipse, die das Polygon minimal umschliesst $\in [0, 1]$
  \item \textbf{Major Axis Length:} $4 \sqrt{\lambda_1}$
  \item \textbf{Minor Axis Length:} $4 \sqrt{\lambda_2}$
  \item \textbf{Perimeter:} Laenge der Seitenlinien des Polygons, kann auch fuer Pixelpolygone berechnet werden (siehe Perimeter Estimator von K. Benkrid mit \emph{4- oder 8-connectivity})
  \item \textbf{Orientation:} Winkel zwischen der X-Achse und der \emph{Major Axis}
  \item \textbf{Oriented Bounding Box Area:} Das kleinste Rechteck, dass die Region umschliesst.\todo{wie bei Pixelpolygonen berechnen? Ist ja eng mit Major Axis verwandt.}
  \item \textbf{Oriented Bounding Box Axis 1:} Die Laenge der laengeren Seite der Oriented Bounding Box
  \item \textbf{Oriented Bounding Box Axis 2:} Die Laenge der kuerzeren Seite der Oriented Bounding Box
\end{itemize}

Daraus koennen weitere Features ermittelt werden:

\begin{itemize}
  \item \textbf{Extent:} $\frac{\text{Area}}{\text{Bounding Box Area}}$
  \item \textbf{Solidity:} $\frac{\text{Area}}{\text{Convex Area}}$
  \item \textbf{Equivalent Diameter:} Durchmesser eines Kreises mit dem Flaecheninhalt von \textbf{Area} $\sqrt{\frac{4 \cdot \text{Area}}{\pi}}$
  \item \textbf{Rectangularity:} $\frac{\text{Area}}{\text{OBB Area}}$
  \item \textbf{Circularity:} $\frac{4\pi \cdot \text{Area}}{\text{Perim} \cdot \text{Perim}}$
  \item \textbf{Compactness:} $\frac{\sqrt{\frac{4}{\pi}\text{Area}}}{\text{OBB Axis 1}}$ (normalisiert Equivalent Diameter)
  \item \textbf{Central moment feature $C^2$:} $\frac{\mu_{20} + \mu_{02}}{\mu_{00}}$ \todo{die naechsten drei gewichtet und/oder ungewichtet?}
  \item \textbf{Centrol moment feature $C^4$:} $\frac{\mu_{40} + \mu_{22} + \mu_{04}}{\mu_{00}}$
  \item \textbf{Elongation:} $\frac{\sqrt{(\mu_{20} - \mu_{02})^2 + \mu^2_{11}}}{\mu_{20} + \mu_{02}}$
\end{itemize}

\subsection{Helligkeit-Features}

\begin{itemize}
  \item \textbf{Max Intensity:} hoechste Helligkeit im Segment
  \item \textbf{Mean Intensity:} durchschnittliche Helligkeit im Segment
  \item \textbf{Min Intensity:} geringste Helligkeit im Segment
  \item \textbf{Weighted Local Centroid:} Zentrum/Schwerpunkt des Helligkeitssegments relativ zur Bounding Box
\end{itemize}

\subsection{Farb-Features}

\begin{itemize}
  \item \textbf{Mean Color:} durchschnittliche Farbe
  \item \textbf{Total Color:} Aufaddierte Farbe aller Pixel im Segment
  \item \textbf{Absolute Difference:} Spannbreite der einzelnen Farbkanaele
\end{itemize}

\subsection{Hole-Features}

\begin{itemize}
  \item \textbf{Filled Area}: Anzahl der Pixel, die das Segment enthaelt, wenn Loecher aufgefuellt werden
  \item \textbf{Number of Holes} oder \textbf{Euler Number:} Anzahl an Loechern im Segment, definiert als $1 - \text{Number of Holes}$ (\emph{4- oder 8-connectivity})
\end{itemize}
