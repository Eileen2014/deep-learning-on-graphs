\chapter{Gedachter Inhalt}

Einleitung:
Motivation
Aufbau der Arbeit

Grundlagen:
Graphen, insbesondere planare Graphen
Mathematische Notationen: Vektor, Matrix, Tensor
Neuronale Netze (Was ist ein CNN, wie ist der Convolution Operator definiert, nicht lineare Aktivierungsfunktion)
Faltung, Faltung im CNN

Graphrepräsentationen von Bildern
Grid
Superpixel
Superpixelalgorithmen
Umwandlung von Kanten von Distanz zu Gauß
Merkmalextraktion (Momente)
Merkmalselektion (Cov, PCA)

Lernen auf Graphen:
Stand der Forschung: Spatial vs Spectral

Spatial:
Patchy
Zentralität
Canonical Labeling
Übertragung auf planare Graphen <- EIGENER ANTEIL (z.B. Grid Spiral)
Komplexität
Vorteile (einfache Architektur)/Nachteile (keine direkte Nachbarschaftsberücksichtigung möglich, keine Graph Coarsening möglich, Vorverarbeitung ist recht teuer und muss Preprocessed werden weil man das nicht über Matrixoperationen ausdrücken kann)

Spectral:
Laplacian, Fourier Transformation
GCN und kGCN (weisfeiler Lehman)
Übertragung auf planare Graphen (Adjazenzpartitionierung) <- EIGENER ANTEIL
Pooling/Coarsening
Komplexität
Vorteile (z.B. Nachbarschaftsberücksichtigung/keine Ordnung nötig)/Nachteile (rotationsinvariant)

Deep Learning auf variabler Input-Menge (SPP)

Augmentierung von Graphen (ist das überhaupt möglich)

Realisierung (Experimente) und Evaluation
Adam-Optimizer
Sparse Tensors
Vorstellung Datensätze (MNIST, PascalVOC, CIFAR-10, ImageNet)
Tensorflow
Dropout L2-Regularisierung

Zusammenfassung und Ausblick
