\chapter{Spektrale Graphentheorie}

\begin{itemize}
  \item \emph{Spektrum} eines Graphen zur Untersuchung seiner Eigenschaften
  \item \emph{algebraische} oder \emph{spektrale Graphentheorie} genannt
  \item als Spektrum eines Graphen bezeichnet man die (nach Größe geordnete) Folge der Eigenwerte $\lambda$ seiner Adjazenzmatrix, d.h. $A \cdot x = \lambda x$ ($x$ Eigenvektoren)
\end{itemize}

Algebraische Methoden sind sehr effektiv bei Graphen, die regulär und symmetrisch sind.
Als \emph{Schleife} wird in der Graphentheorie eine Kante bezeichnet, die einen Knoten mit sich selbst verbindet.
Ein Graph ohne Schleifen wird \emph{schleifenloser} Graph genannt.

Sei $d_v$ der Grad eines Knotens $v$ eines Graphen $G$.
Der \emph{Laplacian} $\mathcal{L}$ eines Graphen ohne Schleifen und Mehrfachkanten ist definiert als

\begin{equation}
  \mathcal{L}(u, v) = \begin{cases}
    d_v, & \text{wenn }u = v\text{,}\\
    -1, & \text{wenn }u\text\ {und }v\ \text{adjazent,}\\
    0, & \text{sonst.}
  \end{cases}
\end{equation}

Damit ist $\mathcal{L} = I - A$.
$\mathcal{L}$ kann normalisiert werden über $\mathcal{L}_{\text{norm}} = T^{-\frac{1}{2}}LT^{-\frac{1}{2}}$, wobei $T$ die Diagonalmatrix beschreibt mit $T(v, v) = d_v$.
Für einen \emph{isolierten} Knoten $v$, d.h.\ $d_v = 0$, gilt die Konvention $T^{-1}(v, v) = 0$.
Ebenso lässt sich $\mathcal{L}_{\text{norm}}$ definieren als

\begin{equation}
  \mathcal{L}_{\text{norm}}(u, v) = \begin{cases}
    1, & \text{wenn }u = v\text{und }d_v \neq 0\text{,}\\
    -\frac{1}{\sqrt{d_u d_v}}, & \text{wenn }u\text\ {und }v\ \text{adjazent,}\\
    0, & \text{sonst.}
  \end{cases}
\end{equation}

Wenn $G$ $k$-regulär ist, d.h.\ $T = \text{diag}(k)$, dann gilt $\mathcal{L}_{\text{norm}} = I - \frac{1}{k}A$.

Da $\mathcal{L}$ symmetrisch ist, sind seine Eigenwerte alle reell und positiv.

Einem gewichtetem ungerichterem Graph $G$ kann eine Gewichtsfunktion $w: V \times V \rightarrow \mathbb{R}$ zugeschrieben werden, sodass $w(u, v) = w(v, u)$ und $w(u, v) \geq 0$.
Falls $\lbrace u, v \rbrace \notin \mathcal{E}$, dann $w(u, v) = 0$.
Damit sind ungewichtete Graphen nur ein Spezialfall bei dem alle Gewichte $0$ oder $1$ sind.
Der Grad $d_v$ eines Knoten $v$ ist dann definiert als

\begin{equation}
  d_v = \sum_u w(u, v).
\end{equation}

Dann gilt

\begin{equation}
  \mathcal{L} = \begin{cases}
    1 - \frac{w(v, v)}{d_v}, & \text{wenn }u = v\text{und }d_v \neq 0\text{,}\\
    -\frac{w(u,v)}{\sqrt{d_u d_v}}, & \text{wenn }u\text\ {und }v\ \text{adjazent,}\\
    0, & \text{sonst.}
  \end{cases}
\end{equation}

Eine Verschrumpfung eines Graphen $G$ kann beschrieben werden über zwei verschiedene Knoten $u$ und $v$ zu einem neuen Knoten $v^*$ mit

\begin{equation}
  w(x,v^*) = w(x, u) + w(x, v)
\end{equation}

und

\begin{equation}
  w(v^*, v^*) = w(u, u) + w(v, v) + 2w(u,v)
\end{equation}

Mit $\lambda_G := \lambda_1$ für einen Graphen $G$, gilt für einen Graphen $H$ der aus $G$ verkleinert wurde

\begin{equation}
  \lambda_G \leq \lambda_H
\end{equation}
