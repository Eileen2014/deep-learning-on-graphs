\section{Beispiel}

Wir betrachten eine einfache $3 \times 3$ Adjazenzmatrix, d.h.\ $|\mathcal{V}| = n = 3$.

\begin{equation}
  A = \begin{pmatrix}
    0 & 1 & 0\\
    1 & 0 & 1\\
    0 & 1 & 0
  \end{pmatrix}
\end{equation}

mit Diagonalmatrix $D = \text{diag}(1, 2, 1)$.

Der Laplacian $\mathcal{L} = D - A$ ist dann

\begin{equation}
  \mathcal{L} = \begin{pmatrix}
    1 & -1 & 0\\
    -1 & 2 & -1\\
    0 & -1 & 1
  \end{pmatrix}
\end{equation}

Nun müssen die Eigenvektoren der Matrix und dessen Eigenwerte bestimmt werden, d.h.\ wir müssen das folgende Eigenwertproblem lösen

\begin{equation}
  \mathcal{L} \cdot \vec{u} = \lambda \cdot \vec{u}
\end{equation}

Wir erhalten $3$ Eigenvektoren und Eigenwerte mit

\begin{equation}
  \lambda_0 = 0, \vec{u}_0 = \frac{1}{\sqrt{3}} \begin{pmatrix}1\\1\\1\end{pmatrix} \approx \begin{pmatrix}0.58\\0.58\\0.58\end{pmatrix},
    \lambda_1 = 1, \vec{u}_1 = \frac{1}{\sqrt{2}} \begin{pmatrix}-1\\0\\1\end{pmatrix} \approx \begin{pmatrix}-0.71\\0\\0.71\end{pmatrix},
      \lambda_2 = 3, \vec{u}_2 = \frac{1}{\sqrt{6}} \begin{pmatrix}1\\-2\\1\end{pmatrix} \approx \begin{pmatrix}0.41\\-0.82\\0.41\end{pmatrix}
\end{equation}

Dann sind $U$, $\Lambda$ und $U^T$ definiert als

\begin{equation}
  U \approx \begin{pmatrix}
    0.58 & -0.71 & 0.41\\
    0.58 & 0 & -0.82\\
    0.58 & 0.71 & 0.41
  \end{pmatrix},
  \Lambda = \begin{pmatrix}
    0 & 0 & 0\\
    0 & 1 & 0\\
    0 & 0 & 3
  \end{pmatrix},
  U^T \approx \begin{pmatrix}
    0.58 & 0.58 & 0.58\\
    -0.71 & 0 & 0.71\\
    0.41 & -0.82 & 0.41
  \end{pmatrix}
\end{equation}

Angenommen wir haben ein Signal $x = {(100, 10, 1)}^T$, dann ist der Wert dieses Signals transformiert in die Fourier Domäne definiert als $\hat x \approx {(64.09, -70.00, 33.07)}^T$.
Führen wir $\hat x$ auf $x$ mittels $U \cdot \hat x$ zurück, erhalten wir korrekterweise $x = {(100, 10, 1)}^T$.

Es gilt $\lambda_{\max} = 3$
Jetzt ist $\mathbf{\tilde \Lambda}$ definiert als
\begin{equation}
  \mathbf{\tilde \Lambda} = \begin{bmatrix}
    -1 & 0 & 0\\
    0 & -\frac{1}{3} & 0\\
    0 & 0 & 1
  \end{bmatrix}
\end{equation}

Wir überprüfen die Approximation durch die Polynome mit $k = 2$:

\begin{equation}
  g_{\mathbf{\theta}}\left(\mathbf{\Lambda}\right) = \begin{pmatrix}
    1\\2\\3
  \end{pmatrix},
  g_{\mathbf{\theta}^{prime}} = \begin{pmatrix}
    wd
  \end{pmatrix}
\end{equation}
