\chapter{Spektrale Graphentheorie}

\begin{itemize}
  \item \emph{Spektrum} eines Graphen zur Untersuchung seiner Eigenschaften
  \item \emph{algebraische} oder \emph{spektrale Graphentheorie} genannt
  \item als Spektrum eines Graphen bezeichnet man die (nach Größe geordnete) Folge der Eigenwerte $\lambda$ seiner Adjazenzmatrix, d.h. $A \cdot x = \lambda x$ ($x$ Eigenvektoren)
\end{itemize}

Algebraische Methoden sind sehr effektiv bei Graphen, die regulär und symmetrisch sind.
Als \emph{Schleife} wird in der Graphentheorie eine Kante bezeichnet, die einen Knoten mit sich selbst verbindet.
Ein Graph ohne Schleifen wird \emph{schleifenloser} Graph genannt.

Sei $d_v$ der Grad eines Knotens $v$ eines Graphen $G$.
Der \emph{Laplacian} $\mathcal{L}$ eines Graphen ohne Schleifen und Mehrfachkanten ist definiert als

\begin{equation}
  \mathcal{L}(u, v) = \begin{cases}
    d_v, & \text{wenn }u = v\text{,}\\
    -1, & \text{wenn }u\text\ {und }v\ \text{adjazent,}\\
    0, & \text{sonst.}
  \end{cases}
\end{equation}

Der Graph Laplacian ist eine Generalisierung des Laplacian auf einem Gitter.

Damit ist $\mathcal{L} = D - A$.
$\mathcal{L}$ kann normalisiert werden über $\mathcal{L}_{\text{norm}} = T^{-\frac{1}{2}}LT^{-\frac{1}{2}}$, wobei $T$ die Diagonalmatrix beschreibt mit $T(v, v) = d_v$.
Für einen \emph{isolierten} Knoten $v$, d.h.\ $d_v = 0$, gilt die Konvention $T^{-1}(v, v) = 0$.
Ebenso lässt sich $\mathcal{L}_{\text{norm}}$ definieren als

\begin{equation}
  \mathcal{L}_{\text{norm}}(u, v) = \begin{cases}
    1, & \text{wenn }u = v\text{und }d_v \neq 0\text{,}\\
    -\frac{1}{\sqrt{d_u d_v}}, & \text{wenn }u\text\ {und }v\ \text{adjazent,}\\
    0, & \text{sonst.}
  \end{cases}
\end{equation}

Der normalisierte Laplacian ist immer noch symmtrisch!

Wenn $G$ $k$-regulär ist, d.h.\ $T = \text{diag}(k)$, dann gilt $\mathcal{L}_{\text{norm}} = I - \frac{1}{k}A$.

Da $\mathcal{L}$ symmetrisch ist, sind seine Eigenwerte alle $\geq 0$ (d.h. $\mathcal{L}$ ist positiv-semidefinit).
Jede Reihen- und Spaltensumme von $\mathcal{L}$ ist $0$.

Einem gewichtetem ungerichterem Graph $G$ kann eine Gewichtsfunktion $w: V \times V \rightarrow \mathbb{R}$ zugeschrieben werden, sodass $w(u, v) = w(v, u)$ und $w(u, v) \geq 0$.
Falls $\lbrace u, v \rbrace \notin \mathcal{E}$, dann $w(u, v) = 0$.
Damit sind ungewichtete Graphen nur ein Spezialfall bei dem alle Gewichte $0$ oder $1$ sind.
Der Grad $d_v$ eines Knoten $v$ ist dann definiert als

\begin{equation}
  d_v = \sum_u w(u, v).
\end{equation}

Dann gilt

\begin{equation}
  \mathcal{L} = \begin{cases}
    1 - \frac{w(v, v)}{d_v}, & \text{wenn }u = v\text{und }d_v \neq 0\text{,}\\
    -\frac{w(u,v)}{\sqrt{d_u d_v}}, & \text{wenn }u\text\ {und }v\ \text{adjazent,}\\
    0, & \text{sonst.}
  \end{cases}
\end{equation}

Bemerke, dass hier Schleifen nicht explizit ausgeschlossen werden!

Eine Verschrumpfung eines Graphen $G$ kann beschrieben werden über zwei verschiedene Knoten $u$ und $v$ zu einem neuen Knoten $v^*$ mit

\begin{equation}
  w(x,v^*) = w(x, u) + w(x, v)
\end{equation}

und

\begin{equation}
  w(v^*, v^*) = w(u, u) + w(v, v) + 2w(u,v)
\end{equation}

Mit $\lambda_G := \lambda_1$ für einen Graphen $G$, gilt für einen Graphen $H$ der aus $G$ verkleinert wurde

\begin{equation}
  \lambda_G \leq \lambda_H
\end{equation}

\section{Spectral Graph Domain}

\begin{itemize}
  \item \emph{Spectral Graph Domain}: Der Raum der Eigenfunktionen von $\mathcal{L}$
  \item Analogon (Nachbildung) einer \emph{Fourier-Transformation} von Funktionen auf gewichteten Graphen
\end{itemize}

Eine beliebige Funktion $f: V \rightarrow \mathbb{R}$ kann als ein Vektor in $\mathbb{R}^n$ gesehen werden.
Dies impliziert eine Ordnung auf den Knoten.
Wir schreiben $f \in \mathbb{R}^n$ für Funktionen auf den Knoten eines Graphen und $f(m)$ für den Wert des $m$ten Knoten.

Dann gilt für eine beliebige Funktion $f \in \mathbb{R}^n$

\begin{equation}
  \mathcal{L}f(x) = \sum_{x~y} w(x, y) \cdot (f(x) - f(y))
\end{equation}

wobei die Summe über $x~y$ die Summierung über alle Knoten $y$ beschreibt, die adjazent zu $x$ sind.

Angenommen $G$ ist als ein reguläres Gitter definiert der Breite und Höhe $M$
Dann hat ein Knoten $v_{x,y}$ genau 4 Nachbarn mit Kantengewicht $\frac{1}{{(\delta w)}^2}$, bei dem $\delta w$ die euklidsche Distanz zwischen zwei Gitterpunkten beschreibt.

Für eine Funktion $f: M \times M \rightarrow \mathbb{R}$ gilt dann:

\begin{equation}
  \mathcal{L}f(x, y) = \frac{4f(x,y) - f(x+1, y) - f(x-1, y) - f(x, y+1) - f(x, y-1)}{{(\delta w)}^2}
\end{equation}

Damit kann ein Signal $f$ mit der Multiplikation mit $\mathcal{L}$ als eine Weiterpropagation von $f$ unter der Berücksichtigung der lokalen Nachbarn verstanden werden (\emph{5-point Stencil}, d.h.\ $\mathcal{L}f \approx - \nabla^2 f$).

\section{Diskrete Fourier Transformation}

$\mathcal{L}$ besitzt genau $n$ orthogonal zueinander stehende Eigenvektoren $\lbrace u_l \rbrace_{l=1}^n \in \mathbb{R}^n$.
Eigenvektoren $u_i$ sind auf $1$ normiert, d.h.\ $||u_i||_2 = 1$.
Diese werden auch \emph{Graph Fourier Modes} genannt.
Diesen sind Eigenwete $\lbrace \lambda_l \rbrace_{l=1}^n \in \mathbb{R}$ zugeordnet, die die \glqq{}Frequenzen\grqq\ bzw.\ das Spektrum des Graphen beschreiben oder visuell betrachtet die Ausdehnung des Raumes, den die Eigenvektoren aufspannen.
Bemerke dass $\lambda_0 = 0$, da für den Eigenvektor $\vec{u_0} = {(1, 1, \ldots, 1)}^T$ gilt, dass $\mathcal{L}\vec{u_0} = 0$.
$\mathcal{L}$ ist diagonalisierbar über $\mathcal{L} = U \Lambda U^T$, wobei $U = [u_1, \ldots, u_n] \in \mathbb{R}^{n \times n}$ die \emph{Fourier Basis} und $\Lambda = \text{diag}([\lambda_0, \ldots, \lambda_n]) \in \mathbb{R}^{n \times n}$.
Die \emph{Fourier Transformation} eines Signals $x \in \mathbb{R}^n$ ist dann definiert als $\hat{x} = U^{T}x$ und die Inverse als $x = U\hat{x}$.

\section{Faltung}

Wir suchen einen Operator $x *_G g$, der eine Faltung zweier Eingangssignale $x, g$ zu einem Ausgangssignal umleitet.
$x$ beschreibt dabei die Knotenattribute und $g$ die Gewichte.

\subsection{Faltung in CNNs}

In der Funktionalanalysis beschreibt die \emph{Faltung} einen mathematischen Operator, der für zwei Funktion $f$ und $g$ eine dirtte Funktion $f * g$ liefert.
Die Faltung kann als ein Produkt von Funktionen vertanden werden.

Anschaulich ist $(f * g)(x)$ der \emph{gewichtete Mittelwert} von $f$, wobei die Gewichtung durch $g$ gegeben ist.

Angenommen wir wollen über einer Matrix mit einem \emph{Filter} falten.
Sei unsere Eingangsmatrix $3 \times 4$ und unsere Filtergröße $2 \times 2$.

Dann gilt zum Beispiel für den Faltungsoperator $*$ in einem Convolutional Neural Network:

\begin{equation}
  \begin{pmatrix}
    1 & 2 & 3 & 1\\
    4 & 5 & 6 & 1\\
    7 & 8 & 9 & 1
  \end{pmatrix} * \begin{pmatrix}
    1 & 1\\
    1 & 1
  \end{pmatrix} = \begin{pmatrix}
    12 & 16 & 11\\
    24 & 28 & 17
  \end{pmatrix}
\end{equation}

$f: 3 \times 4 \rightarrow \mathbb{R}$ und $g: 2 \ times 2 \rightarrow \mathbb{R}$, dann ist $*$ definiert als

\begin{equation}
  (f * g)(x, y) = \sum_{x_i \in [x, x+1]\\y_i \in [y, y+1]} f(x_i, y_i)g(x-x_i, y-y_i)
\end{equation}

\subsection{Faltung auf Graphen}

Da wir keinen Translationsoperator auf der Domäne der Knoten $x$ beschreiben können, müssen wir unseren Faltungsoperator in der Fourier-Domäne beschreiben.
Dafür wandeln wir unsere Knotenmenge $x$ zuerst in $\hat x$ um.

Wir definieren $*_G$ in der Fouier-Domäne als

\begin{equation}
  x *_G g = U \cdot (U^T \cdot x \odot \hat g)
\end{equation}

wobei $\odot(A, B) = (a_{ij} \cdot b_{ij})$ die elementweise Multiplikation bzw.\ das \emph{Hadamard-Produkt}.

Das Hadamard-Produkt löst sich auf, wenn $\hat g$ als eine Diagonalmatrix repräsentiert wird. Dann gilt

\begin{equation}
  x *_G g = U \begin{pmatrix}
    \hat g(\lambda_0) & \cdots & 0\\
    0 & \cdots & \hat g(\lambda_n)
  \end{pmatrix}U^T x = U \hat g(\Lambda) U^T x
\end{equation}

Dann beschreibt $\hat g(\Lambda) = \text{diag}(\theta)$ eine Gewichtsfunktion mit $n$ Variablen, $\theta \in \mathbb{R}^n$.
Damit ist die Faltung bzw.\ die Gewichtung abhängig von der Input-Größe $n$, was extrem schlecht ist.

\subsection{Offene Fragen}

\begin{itemize}
  \item Wie erklärt sich noch einmal der normalisierte Laplacian?
  \item Warum wird $\hat g$ als Diagonalmatrix repräsentiert?
  \item Wie kommt die Convolution zustande mit dem $*$ Operator?
  \item Was passiert bei gerichteten Graphen???? Wir haben keinen symmetrischen und insbesondere keinen positiv definiten
\end{itemize}

\subsection{Beispiel}

Wir betrachten eine einfache $3 \times 3$ Adjazenzmatrix, d.h.\ $|\mathcal{V}| = n = 3$.

\begin{equation}
  A = \begin{pmatrix}
    0 & 1 & 0\\
    1 & 0 & 1\\
    0 & 1 & 0
  \end{pmatrix}
\end{equation}

mit Diagonalmatrix $D = \text{diag}(1, 2, 1)$.

Der Laplacian $\mathcal{L} = D - A$ ist dann

\begin{equation}
  \mathcal{L} = \begin{pmatrix}
    1 & -1 & 0\\
    -1 & 2 & -1\\
    0 & -1 & 1
  \end{pmatrix}
\end{equation}

Nun müssen die Eigenvektoren der Matrix und dessen Eigenwerte bestimmt werden, d.h.\ wir müssen das folgende Eigenwertproblem lösen

\begin{equation}
  \mathcal{L} \cdot \vec{u} = \lambda \cdot \vec{u}
\end{equation}

Wir erhalten $3$ Eigenvektoren und Eigenwerte mit

\begin{equation}
  \lambda_0 = 0, \vec{u}_0 = \frac{1}{\sqrt{3}} \begin{pmatrix}1\\1\\1\end{pmatrix} \approx \begin{pmatrix}0.58\\0.58\\0.58\end{pmatrix},
    \lambda_1 = 1, \vec{u}_1 = \frac{1}{\sqrt{2}} \begin{pmatrix}-1\\0\\1\end{pmatrix} \approx \begin{pmatrix}-0.71\\0\\0.71\end{pmatrix},
      \lambda_2 = 3, \vec{u}_2 = \frac{1}{\sqrt{6}} \begin{pmatrix}1\\-2\\1\end{pmatrix} \approx \begin{pmatrix}0.41\\-0.82\\0.41\end{pmatrix}
\end{equation}

Dann sind $U$, $\Lambda$ und $U^T$ definiert als

\begin{equation}
  U \approx \begin{pmatrix}
    0.58 & -0.71 & 0.41\\
    0.58 & 0 & -0.82\\
    0.58 & 0.71 & 0.41
  \end{pmatrix},
  \Lambda = \begin{pmatrix}
    0 & 0 & 0\\
    0 & 1 & 0\\
    0 & 0 & 3
  \end{pmatrix},
  U^T \approx \begin{pmatrix}
    0.58 & 0.58 & 0.58\\
    -0.71 & 0 & 0.71\\
    0.41 & -0.82 & 0.41
  \end{pmatrix}
\end{equation}

Angenommen wir haben ein Signal $x = {(100, 10, 1)}^T$, dann ist der Wert dieses Signals transformiert in die Fourier Domäne definiert als $\hat x \approx {(64.09, -70.00, 33.07)}^T$.
Führen wir $\hat x$ auf $x$ mittels $U \cdot \hat x$ zurück, erhalten wir korrekterweise $x = {(100, 10, 1)}^T$.

Ich habe absolut keine Ahnung was diese Sachen aussagen sollen :D:D

\section{Chebyshev Polynome}

\section{Probleme}

Rotationsinvariant

\section{Pfadlänge}

wenn $d_G(m,n) > k$, dann ${(L^k)}_{m, n} = 0$
(normalisiert sowie unnormalisiert (siehe Wavelet Lemma 5.4))
