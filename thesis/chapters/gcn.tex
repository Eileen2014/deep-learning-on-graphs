\chapter{Graph Convolutional Networks}

\begin{equation}
  H^{(l+1)} = f(H^{(l)}, A)
\end{equation}

\begin{equation}
  f(H^{(l)}, A) = \sigma(AH^{(l)}W^{(l)})
\end{equation}

\begin{equation}
  D_{ii} = \sum_j A_{ij}
\end{equation}

Für die Potenz $x \in \gls{R}$ einer Diagonalmatrix $D \in \gls{R}^{N \times N}$ gilt:

\begin{equation}
  D^x = \begin{pmatrix}
    d_{11} & 0 & \cdots & 0\\
    0 & d_{22} & \cdots & 0\\
    \vdots & \vdots & \ddots & \vdots\\
    0 & 0 & \cdots & d_{nn}\\
  \end{pmatrix}^x = \begin{pmatrix}
    d_{11}^x & 0 & \cdots & 0\\
    0 & d_{22}^x & \cdots & 0\\
    \vdots & \vdots & \ddots & \vdots\\
    0 & 0 & \cdots & d_{nn}^x\\
  \end{pmatrix}
\end{equation}

\section{Erweiterung für mehrere Kantenattribute}

Graph Convolutional Networks berücksichtigen nur eine Adjazenzmatrix.
Das bedeutet insbesondere, dass ein Graph nur über ein Kantenattribut verfügen kann.
Das ist für ungewichtete Graphen die Markierung einer Kante ($a_{ij} \in \lbrace 0, 1 \rbrace$) oder für gewichte Graphen das Gewicht einer Kante ($a_{ij} \in \gls{R+}$).
Eine Menge von Kantenattributen kann über mehrere Adjazenzmatrizen definiert werden.
Damit ist es ebenfalls möglich unterschiedliche Kanten für unterschiedliche Attribute zu definieren.

Eine Menge von Adjazenzmatrizen $\mathcal{A} = \lbrace A_1, A_2, \ldots, A_m \rbrace$ mit $A_i \in \gls{R}^{n \times n}$ beschreibt damit eine Menge von $m$ Graphen über der gleichen Knotenmenge $\mathcal{V}$ mit Kardinalität $n$.

$\mathcal{A} \in \gls{R}^{m \times n \times n}$ kann zu einer zweidimensionalen Matrix $A \in \gls{R}^{m \cdot n \times n}$ geglättet werden.

Dann ist $A \cdot H^{(l)} \in \gls{R}^{m \cdot n \times d}$.
Reshape zu $\gls{R}^{n \times m \cdot d}$ und Gewichtsmatrix $G \in \gls{R}^{m \cdot d \times x}$.

\subsection{Übertragung auf eingebettete Graphen}

Graphknoten haben im Allgemeinen keine Position oder Lage im Raum.
Knoten, die Regionen in einer vorhandenen Segmentierung darstellen, haben jedoch offensichtlich eine gewisse Lage im Raum, die zum Beispiel über das Zentrum der Region definiert werden kann.
Diese Information ist vorhanden und wichtig und sollte demnach auch nicht verloren gehen.
Anstatt diese lokal im Knoten zu speichern, bietet es sich eher an diese Information in den Kanten zu speichern um eine bessere Faltung zu garantieren.
Die euklische Distanz zwischen zwei benacharten Regionszentren wahrt zwar die Information der Distanz zweier Knoten zueinander, verliert aber die Information der Position zweier Knoten zueinander.
Es bietet sich daher an, die horizontalen und vertikalen Abstände in einer Koordinate an den Kanten zu speichern.
Es ist zu beachten, dass wir dadurch zu einem gerichteten Graphen übergehen, bei dem jede Kante von $v$ nach $w$ auch eine Kante von $w$ nach $v$ besitzt.

Wir haben damit zwei Adjazenzmatrizen.
Da Graph Convolutional Networks nicht mit negativen Gewichten funktionieren, müssen wir negative Koordinaten in eine weitere Adjazenzmatrix schreiben.
Wir gelangen damit zu vier Adjazenzmatrizen, die die Verbindungen von einem Knoten beschreibt, die links, rechts, oben oder unten zu ihm liegen.
Wir definieren diese Adjazenzmatrizen respektive als $A_{\text{links}}, A_{\text{rechts}}, A_{\text{oben}}$ und $A_{\text{unten}}$.

Kantenattribute bzw.\ Positionen von Knoten sollten skalierungsinvariant gespeichert werden.
Dafür werden die Abstände auf den Einheitskreis gemappt, wobei die längste Distanz eines Knotens auf dem Einheitskreis liegt (vgl. Abbildung~\ref{einheitskreis}).

\begin{figure}
\begin{tikzpicture}[]
  \draw[dashed] (0, 0) circle (4);

  \tikzstyle{node}=[circle,draw, minimum width=5pt, inner sep=0pt, fill=black]
  \tikzstyle{root}=[fill=red]

  \node[node, root] (root) at (0,0) {};
  \node[node, label={[fill=white]150:$(-1, 2) \rightarrow (-.25,.5)$}] (a) at (-1, 2) {};
  \node[node, label={[fill=white]$(4, 0) \rightarrow (1, 0)$}] (b) at (4, 0) {};
  \node[node, label={[fill=white]$(2, -2) \rightarrow (.5,-.5)$}] (c) at (2, 2) {};
  \node[node, label={[fill=white]150:$(-3, -1) \rightarrow (-.75, -.25)$}] (d) at (-3, -1) {};
  \node[node, label={[fill=white]50:$(1, -3) \rightarrow (.25, -.75)$}] (e) at (1, -3) {};

  \path (root) edge (a);
  \path (root) edge (b);
  \path (root) edge (c);
  \path (root) edge (d);
  \path (root) edge (e);
\end{tikzpicture}
\caption{Abbildung der lokalen Nachbarschaftsknoten auf den Einheitskreis.}
\label{einheitskreis}
\end{figure}

Für die Anwendung auf das Graph Convolutional Network müssen die Gewichte aller Adjazenzmatrizen $a_{xij} \in [0, 1]$ invertiert werden, damit nähere Knoten einen größeren Einfluss haben.
Ebenso müssen \emph{Self Loops} für alle Knoten hinzugefügt werden.
Wir definieren unsere Adjazenzmatrix $\tilde A \in \mathbb{R}^{N \times N}$ aus einer Adjazenzmatrix $A \in \mathbb{R}^{N \times N}$ dann über

\begin{equation}
  \tilde A_{ij} = \begin{cases}
    1, & \text{falls }i=j\text{,}\\
    {(a_{ij}+1)}^{-1}, & \text{falls }a_{ij} \neq 0\text{,}\\
    0, & \text{sonst.}
  \end{cases}
\end{equation}
