\section{Der Laplacian und seine Eigenwerte}

Der \emph{kombinatorische Laplacian} \gls{L} eines Graphen \gls{G} ist definiert als $\gls{L} = \gls{D} - \gls{A}$~\cite{Chung}.
$\gls{L}$ ist eine symmetrische Matrix.

Der \emph{normalisierte Laplacian} \gls{Lnorm} ist definiert als $\gls{Lnorm} = \gls{D}^{-\frac{1}{2}} \gls{L} \gls{D}^{-\frac{1}{2}}$~\cite{Chung}.
Es gilt die Konvention, dass ${\left(\gls{D}^{-\frac{1}{2}}\right)}_{ii} = 0$ falls $\gls{D}_{ii} = 0$ (d.h.\ $v_i$ ist isolierter Knoten)

Für verbundene Graphen gilt damit $\gls{Lnorm} = \gls{I} - \gls{D}^{-\frac{1}{2}} \gls{A} \gls{D}^{-\frac{1}{2}}$~\cite{Chung}.
Jeder Eintrag der Diagonalen von \gls{Lnorm} ist damit $1$.
\gls{Lnorm} ist weiterhin symmetrisch, das wäre bei einer Normierung der Form $\gls{D}^{-1}\gls{L}$ nicht der Fall.

\gls{L} und \gls{Lnorm} sind keine ähnlichen Matrizen.
Insbesondere sind ihre Eigenvektoren unterschiedlich.
Die Nutzung von \gls{L} oder \gls{Lnorm} ist damit abhängig von dem Problem, welches man betrachtet.~\cite{Hammond}.

Wir schreiben \gls{Lboth} wenn die Wahl des Laplacian \gls{L} oder \gls{Lnorm} irrelevant ist.

\subsection{Visuelle Interpretation des Laplacian}

\begin{itemize}
  \item diskrete Analogie des $\nabla^2$ Operators
  \item man nimmt eine Funktion und approximiert sie mit Hilfe eines Graphen, so dass Knoten, die dichter beieinander liegen eine größere zweite Ableitung besitzen.
\end{itemize}


$\nabla^2 f = \nabla \cdot \nabla f$

Die Divergenz eines Vektorfeldes ist ein Skalarfeld, das an jedem Punkt angibt, wie sehr die Vektoren in einer kleinen Umgebung des Punktes auseinanderstreben.

The Laplace operator measures how much a function differs at a point from the average of the values of the function over small spheres centered at that point. As it turns out, the Laplacian of a graph does something completely analogous: namely, it measures how much a function on a graph differs at a vertex from the average of the values of the function over the neighbors of the vertex.

Im $n$-dimensionalen euklischen Raum
\begin{equation}
  \nabla^2f = \sum_{i=1}^n \frac{\partial^2f}{\partial x^2_k}
\end{equation}
in einer Dimension reduziert sich der Laplace-Operator auf die zweite Ableitung $\nabla^2 f = f^{\prime\prime}$.

Der \emph{diskrete Laplace-Operator} ist eine Analogie zum diskreten Laplace-Operator, der finite Differenzen $x \pm h$ zur Approximation von $\nabla^2 f$ nutzt
\todo{Laplace-Operator}
\todo{diskreter Laplace-Operator} Approximation des Laplace-Operators für finite Elemente

Sei $f \colon \gls{V} \to \gls{R}$ eine Funktion auf den Knoten eines Graphen.
$f$ kann ebenso als Vektor $\ve{f} \in \gls{R}^n$ betrachtet werden mit der Ordnung der Knoten, die die Adjazenzmatrix vorgibt.

Dann gilt für \gls{Lboth}, dass
\begin{equation}
  {\left(\gls{Lboth}\ve{f}\right)}_i = \sum^n_{\substack{j=0\\j \neq i}} -\gls{Lboth}_{ij} \left(\ve{f}_i - \ve{f}_j\right)
\end{equation}

Für einen Graphen, der ein reguläres Gitter aufspannt mit gleichen Kantengewichten $\frac{1}{h^2} \in \gls{R}$ gilt für einen Knoten an Position $\left(x, y\right)$:
Abusing the index notation
\begin{equation}
  {\left(\gls{L}\ve{f}\right)}_{x,y} = \frac{4\ve{f}_{x,y} - \ve{f}_{x+1,y} - \ve{f}_{x-1,y} - \ve{f}_{x,y+1} - \ve{f}_{x,y-1}}{h^2}
\end{equation}
beschreibt die \emph{5-Punkte-Stern} Approximation $-\nabla^2 f$.
mit $\nabla^2 f$ definiert auf den fünf Punkten $\left\{\left(x,y\right), \left(x+h,y\right), \left(x-h,y\right), \left(x,y+h\right),\left(x,y-h\right)\right\}$.
\begin{equation}
  \gls{Lboth}f \approx - \nabla^2 f
\end{equation}

\todo{vertauschen, erst beispiel auf gitter, dann analog zu graphen gitter}

Damit kann der Graph Laplacian als eine Generalisierung des diskreten Laplacian auf einem Gitter verstanden werden.

Eigenwerte und Eigenvektoren werden benutzt, um zu verstehen was passiert, wenn wir einen Operator (hier \gls{Lboth}) mehrfach auf einen Vektor $\ve{x}$ anwenden (hier Merkmal auf den Knoten).

Wir können \ve{x} als Linearkombination der \emph{Eigenbasis} schreiben mit
\begin{equation}
  \ve{x} = \sum_i c_i \gls{eiv}_i
\end{equation}
und berechnen dann
\begin{equation}
  \gls{Lboth}^k \ve{x} = \sum_i c_i \gls{Lboth}^k \gls{eiv}_i = \sum_i = c_i \gls{lambda}_i \gls{Lboth}^{k-1} \gls{eiv}_i = \sum_i c_i \gls{lambda}_i^k \gls{eiv}_i
\end{equation}
Wenn wir einen Operator haben, der einen Graphen beschreibt, dann können Eigenschaften dieses Operators und damit des Graphen selber durch dessen Eigenwerte und Eigenvektoren beschrieben werden.

\subsection{Eigenschaften}

\gls{Lboth} ist eine reell symmetrische, schwach diagonaldominante Matrix und damit insbesondere positiv semidefinit.

$\gls{Lboth} \in \gls{R}^{n \times n}$ hat genau $n$ Eigenwerte ${\left\{\gls{lambda}_i\right\}}_{i = 1}^n \in \gls{R+}$ mit $\gls{lambda}_i \leq \gls{lambda}_{i+1}$.

Anzahl der Eigenvektoren gleich Null ist die Anzahl an Komponenten, die ein Graph besitzt.

Insbesondere sind jede Reihen- und Spaltensumme von $\gls{Lboth}$ ist $0$, d.h.\ $\sum_j \gls{Lboth}_{ij} = 0$ und $\sum_j \gls{Lboth}_{ji} = 0$ für alle $i \in \left\{1, \ldots, n\right\}$.
Insbesondere gilt $\gls{lambda}_1 = 0$, da $\gls{eiv}_1 = \frac{1}{\sqrt{n}}{\left[1, \ldots, 1\right]}^{\top} \in \gls{R}^n$ Eigenvektor von \gls{Lboth} mit $\gls{Lboth}\gls{eiv}_1 = 0$.\\

$0 = \gls{lambda}_1 < \gls{lambda}_2 \leq \cdots \leq \gls{lambda}_n$ wenn Graph verbunden.\todo{quelle}\\

\todo{was sagt $\gls{lambda}_2$ aus?}
Für einen Graphen \gls{G} definieren wir $\gls{lambda}_{\gls{G}} := \gls{lambda}_2$ und $\gls{lambda}_{\max} := \gls{lambda}_n$
Für \gls{Lnorm} gilt $\gls{lambda}_{\max} \leq 2$\todo{quelle}

Für $\gls{Lboth}^{k}$ mit $k \in \gls{N}$ gilt ${\left(\gls{Lboth}^k\right)}_{ij} = 0$ genau dann, wenn $\gls{s}\left(v_i, v_j\right) > k$~\cite{Hammond}.
Damit beschreibt $\gls{Lboth}^k$ bildlich gesprochen die Menge an Knoten, die maximal $k$ Kanten entfernt liegen.

Eine Verschrumpfung eines Graphen \gls{G} kann beschrieben werden über zwei verschiedene Knoten $u$ und $v$ zu einem neuen Knoten $v^*$ mit
\begin{align}
  \gls{w}\left(x,v^*\right) &= \gls{w}\left(x, u\right) + \gls{w}\left(x, v\right)\\
  \gls{w}\left(v^*, v^*\right) &= \gls{w}\left(u, u\right) + \gls{w}\left(v, v\right) + 2\gls{w}\left(u,v\right)
\end{align}

Für einen Graphen \gls{G} gilt für einen Graphen $H$, der aus \gls{G} verkleinert wurde~\cite{Chung},
\begin{equation}
  \gls{lambda}_{\gls{G}} \leq \gls{lambda}_H
\end{equation}
