\section{Eigenwerte und Eigenvektoren reell symmetrischer Matrizen}

$\ma{M}\gls{eiv} = \gls{lambda}\gls{eiv}$\\
Zu einem Eigenwert $\gls{lambda}$ gibt es unendlich viele (skalierte) Eigenvektoren \gls{eiv}.
Wir definieren einen Eigenvektor \gls{eiv} dann eindeutig über $\left\|\gls{eiv}\right\|_2 = 1$.
Sei \ma{M} weiterhin reell symmetrisch, d.h. $\ma{M} = \ma{M}^{\top} \in \gls{R}^{n \times n}$.
Dann gilt für zwei unterschiedliche Eigenvektoren $\gls{eiv}_1$ und $\gls{eiv}_2$, dass $\gls{eiv}_1 \gls{ortho} \gls{eiv}_2$.
\ma{M} hat dann genau $n$ reelle Eigenwerte mit ${\left\{\gls{lambda}_i\right\}}_{i=1}^n$.
Wir definieren $\gls{Lambda} = \gls{diag}\left(\left[\gls{lambda}_1, \ldots, \gls{lambda}_n\right]\right)$.

Wir definieren die orthogonale Matrix $\gls{Eiv} = \left[\gls{eiv}_1, \ldots, \gls{eiv}_n\right] \in \gls{R}^{n \times n}$.
Dann gilt $\ma{M}\gls{Eiv} = \gls{Eiv}\gls{Lambda}$.
und insbesondere ist \ma{M} diagonalisierbar über
\begin{equation}
  \ma{M} = \ma{M}\gls{Eiv}\gls{Eiv}^{\top} = \gls{Eiv}\gls{Lambda}\gls{Eiv}^{\top}
\end{equation}

mit $\gls{Eiv}\gls{Eiv}^{\top} = \gls{I}$.

Damit gilt insbesondere für symmetrisch reelle Matrizen $\ma{M}$, $k \in \gls{N}$
\begin{equation}
  \ma{M}^k = {\left(\gls{Eiv}\gls{Lambda}\gls{Eiv}^{\top}\right)}^k = \gls{Eiv}\gls{Lambda}^k\gls{Eiv}^{\top}
\end{equation}

Diesen Zusammenhang kann man sicht leicht erklären, wenn man die Potenz ausschreibt:
\begin{equation}
  {\left(\gls{Eiv}\gls{Lambda}\gls{Eiv}^{\top}\right)}^k = \gls{Eiv}\gls{Lambda}\gls{Eiv}^{\top}\gls{Eiv}\gls{Lambda}\gls{Eiv}^{\top}\prod^{k-2}_{i=1} \gls{Eiv}\gls{Lambda}\gls{Eiv}^{\top} = \gls{Eiv}\gls{Lambda}^2\gls{Eiv}^{\top} \prod^{k-2}_{i=1} \gls{Eiv}\gls{Lambda}\gls{Eiv}^{\top} = \gls{Eiv}\gls{Lambda}^k \gls{Eiv}^{\top}
\end{equation}

Falls \ma{M} weiterhin \emph{schwach diagonaldominant} ist, d.h
\begin{equation}
  \sum_{j=1}^n \left|\ma{M}_{ij}\right| \leq \left|\ma{M}\right|_{ii}
\end{equation}
für alle $i \in \left\{1, \ldots, n\right\}$ sind ihre Eigenwerte $\lambda_i \in \gls{R+}$ positiv reell und wir können auf diesen eine Ordnung definieren mit $0 \leq \gls{lambda}_1 \leq \cdots \gls{lambda}_n$.
Insbesondere ist \ma{M} dann \emph{positiv-semidefinit}, das bedeutet
\begin{equation}
  \ve{x}^{\top}\ma{M}\ve{x} \geq 0
\end{equation}
