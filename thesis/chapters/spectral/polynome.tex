\section{Polynomielle Approximation}

\begin{itemize}
  \item not localized in Space \todo{was bedeutet das?}
  \item \todo{was bedeutet $k$-localized?}
  \item Learning Komplexität linear zu der Dimension der Daten $n$
\end{itemize}

Wenn wir $\hat g\left(\gls{lambda}_i\right)$ als ein Polynom vom Grad $k$ approximieren, d.h.\

\begin{equation}
  \hat g^{\prime}(\gls{lambda}_i) = \sum_{j=0}^k c_j \gls{lambda}_i^j
\end{equation}
bzw.
\begin{equation}
  \hat g^{\prime}\left(\gls{Lambda}\right) = \sum_{j = 0}^k c_j \gls{Lambda}^j
\end{equation}
mit Koeffizienten $c_0, \ldots, c_k \in \gls{R}$, dann lässt sich der Filterungsprozess in der Spectral Graph Domain erstaunlich gut auch in der Knotendomäne interpretieren.

Der Term $\gls{Eiv}\hat g^{\prime}\left(\gls{Lambda}\right)\gls{Eiv}^{\top}$ kann dann weiter vereinfacht werden:
\begin{equation}
  \hat g^{\prime}\left(\gls{Lboth}\right) =: \gls{Eiv}\hat g^{\prime}\left(\gls{Lambda}\right)\gls{Eiv}^{\top} = \gls{Eiv} \sum_{j=0}^k c_j \gls{Lambda}^j \gls{Eiv}^{\top} = \sum_{j=0}^k c_j \gls{Eiv}\gls{Lambda}^j\gls{Eiv}^{\top} = \sum_{j=0}^k c_j \gls{Lboth}^j
\end{equation}

Das sieht man leicht mit der Beziehung $\gls{Lboth}^k = {\left(\gls{Eiv}\gls{Lambda}\gls{Eiv}^{\top}\right)}^k = \gls{Eiv}\gls{Lambda}^k\gls{Eiv}^{\top}$.\todo{bereits weiter oben definiert}

Dann ist die Faltung definiert als
\begin{equation}
  \ve{f}_{\mathrm{out}} = \hat g^{\prime}\left(\gls{Lboth}\right)\ve{f}_{\mathrm{in}}
\end{equation}
bzw.
\begin{equation}
  f_{\mathrm{out}}\left(v_i\right) = \sum_{j=0} f_{\mathrm{in}}\left(v_j\right) {\left(\hat g^{\prime}\left(\gls{Lboth}\right)\right)}_{ij}
\end{equation}
Wir erinnern uns, dass ${\left(\gls{L}^l\right)}_{ij} = 0$, wenn die kürzeste Pfadlänge $\gls{s}\left(v_i, v_j\right) > l$ von $v_i$ zu $v_j$, d.h.\ die minimale Anzahl an Kanten, größer ist als $l$.

Damit kann die Faltung in der Knotenmenge intuitiv beschrieben werden als eine gewichtete Aufsummierung bzw.\ lineare Kombination der Knotensignale in einer $k$-lokalisierten\todo{definieren} Nachbarschaft um einen Knoten $v_i$
\begin{equation}
  f_{\mathrm{out}}\left(v_i\right) = f_{\mathrm{in}}\left(v_i\right) {\left(\hat g^{\prime}\left(\gls{Lboth}\right)\right)}_{ii} + \sum_{v_j \in \mathcal{N}\left(v_i, k\right)} f_{\mathrm{in}}\left(v_j\right) {\left(\hat g^{\prime}\left(\gls{Lboth}\right)\right)}_{ij}
\end{equation}
