\section{Diskreter geometrischer Kotangens-Laplacian}

\begin{equation}
  \gls{w}\left(v_i, v_j\left) = \frac{\cot\left(\alpha\left(v_i, v_j\right) + \beta\left(v_i, v_j\right)\right)}{2}
\end{equation}
und Grad eines Knoten ist das \emph{Voronoi-Gebiet}
\begin{equation}
  \gls{d}\left(v_i\right) = \gls{a}\left(v_i\right)
\end{equation}
also die Fläche des Voronoi-Gebietes um einen Knoten~\cite{Reuter}.

Winkel $\alpha$ und $\beta$ beschreiben die Winkel, deren Kantenenden von $v_i$ und $v_j$ aufgespannt werden.
Damit ist $\gls{L}$ symmetrisch, denn in der Rückrichtung vertauschen sich nur die Werte von $\alpha$ und $\beta$.

