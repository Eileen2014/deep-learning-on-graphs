\section{Graphentheorie}

Graph Tupel $\gls{G} = \left(\gls{V}, \gls{E}\right)$\\
$\gls{V} = {\left\{ v_i \right\}}^n_{i=1}$\\
$\left| \gls{V} \right| = n < \infty$\\
Merkmalsfunktion $f_G \colon \gls{V} \to \gls{R}^m$\\
wenn nicht explizit aufgeführt, dann bla bla wir $f_{\gls{G}} \colon \gls{V} \to \gls{R}$\\
Umschreibung in Tensor/Dense Matrix\\
$\gls{E} \subseteq \gls{V} \times \gls{V}$\\
Falls $\left( u, v \right) \in \gls{E}$, dann sind $u$ und $v$ adjazent und wir schreiben $u \gls{adj} v$\\
Gewichtsfunktion $\gls{w} \colon \gls{V} \times \gls{V} \to \gls{R+}$\\
ungewichtet: $\gls{w} \colon \gls{V} \times \gls{V} \to \left\{ 0, 1 \right\}$\\
Falls $\left( u, v \right) \notin \gls{E}$, dann $\gls{w}\left(u, v\right) = 0$\\
Im ungewichteten Fall ist Gewichtsfunktion implizit durch $\gls{E}$ gegeben\\

ungerichtet:
$u \gls{adj} v$ genau dann, wenn $v \gls{adj} u$ und
\begin{equation}
  \gls{w}\left(u, v\right) = \gls{w}\left(v, u\right)
\end{equation}
Fordern wir für den Verlauf dieser Arbeit (also keine gerichteten Graphen)

Als \emph{Schleife} wird eine Kante bezeichnet, die einen Knoten mit sich selbst verbindet, d.h.\ $\gls{w}\left(v, v\right) > 0$.
Ein Graph ohne Schleifen wird \emph{schleifenloser Graph} genannt.
Für den weiteren Verlauf dieser Arbeit fordern wir schleifenlose Graphen.\\

Adjazenzmatrix $\gls{A} \in \gls{R+}^{n \times n}$ eines Graphen $\gls{G}$ mit $\gls{A}_{ij} = \gls{w}\left(v_i, v_j\right)$\\
Wir sagen ein Knoten $v_i$ hat Position $i$ in $\gls{A}$.
Umschreibung in Sparse Matrix/Tensor\\

\gls{G} ist eindeutig definiert durch \gls{A} und $f_G$.

Der \emph{Grad} eines Knotens $v$ ist die Anzahl der Knoten, die adjazent zu ihm sind, d.h.
\begin{equation}
  \gls{degree}\left(v\right) = \sum_{v \gls{adj} u} 1
\end{equation}
Im Falle von gewichteten Graphen wird der Grad eines Knotens von $v$ auch oft über
\begin{equation}
  \gls{d}\left(v_i\right) = \sum^n_{j=1} \gls{A}_{ij}
\end{equation}
definiert.
Die unterschiedliche Notation macht deutlich, wann wir welchen Grad eines Knotens meinen.

Die Gradmatrix $\gls{D} \in \gls{R+}^{n \times n}$ eines Graphen \gls{G} ist definiert als Diagonalmatrix
\begin{equation}
  \gls{D} = \gls{diag}\left( {\left[ \gls{d}\left(v_1\right), \ldots, \gls{d}\left(v_n\right) \right]}^{\top} \right)
\end{equation}
Umschreibung in Sparse Matrix/Tensor

% Ein Knoten $v \in \gls{V}$ eines Graphen \gls{G} heißt genau dann \emph{isoliert}, wenn $\gls{degree}\left(v\right) = 0$.\\

Ein Graph heißt \emph{$k$-regulär} falls $\gls{degree}\left(v_i\right) = k$ für alle $1, \ldots, n$.

Ein \emph{ebener Graph} ist eine konkrete Darstellung eines Graphen auf der zweidimensionalen Ebene $\gls{R}^2$.
Jedem Knoten $v$ ist eine Positionsfunktion $\gls{p} \colon \gls{V} \to \gls{R}^2$ zugeordnet, die die Position eines Knotens auf der Ebene eindeutig definiert.

Ein \emph{Weg} ist eine Folge von Knoten $\left( v_{x\left(1\right)}, v_{x\left(2\right)}, \ldots, v_{x\left(k\right)} \right)$, sodass $v_{x\left(i\right)} \gls{adj} v_{x\left(i+1\right)}$ für alle $1 \leq i < k$ mit Länge $k$, wobei $x \colon \left\{ 1, \ldots, n \right\} \to \left\{ 1, \ldots, n \right\}$ eine Permutation auf der Anzahl der Knoten.
\todo{das stimmt noch nicht ganz, gehen ja auch mehrere Knoten}

Ein \emph{Pfad} ist ein Weg, sodass $v_{x\left(i\right)} \neq v_{x\left(i+1\right)}$.
Im Kontext von schleifenlosen Graphen sind die Begriffe Weg und Pfad äquivalent.
Wir schreiben $\gls{s}\left(u, v\right)$ einer Funktion $\gls{s} \colon \gls{V} \times \gls{V} \to \gls{N}$ für die Länge des kürzesten Pfades von $u$ nach $v$.

Ein Graph ist \emph{verbunden}, falls es von jedem Knoten $u$ einen Weg zu jedem Knoten $v$ gibt.
Für den weiteren Verlauf dieser Arbeit fordern wir, dass $G$ verbunden ist.\

In Graphen mit \emph{Mehrfachkanten}, auch \emph{Multigraphen} genannt, können zwei Knoten durch mehrere Kanten verbunden sein.
Multigraphen lassen sich als Tensor über einen Vektor von Adjazenzmatrizen $\left[\gls{A}_1, \ldots, \gls{A}_m\right] \in \gls{R+}^{m \times n \times n}$ schreiben.
