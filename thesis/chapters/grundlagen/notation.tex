\section{Notationen}

$\text{diag} \colon \mathbb{R}^n \to \mathbb{R}^{n \times n}$ einen Vektor $\mathbf{d}$ in Diagonalform $\mathbf{D}$ bringt mit $\mathbf{D}_{ii} = \mathbf{d_i}$ und $\mathbf{D}_{ij} = 0$ für $i \neq j$.
Die Inverse $\text{diag}^{-1} \colon \mathbb{R}^{n \times n} \to \mathbb{R}^n$ liefert zu einer beliebigen Diagonalmatrix $\mathbf{D}$ dessen Diagonalvektor $\mathbf{d}$.
brauch ich glaub ich nicht mehr

Wir erlauben, dass wir eine eindimensionale Funktion $f \colon \mathbb{R} \to \mathbb{R}$ auch elementweise auf Vektoren, Matrizen bzw.\ Tensoren anwenden dürfen.

Identitätsmatrix $\mathbf{I}$
